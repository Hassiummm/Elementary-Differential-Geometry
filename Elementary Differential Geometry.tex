% chktex-file 1 chktex-file 8 chktex-file 9 chktex-file 12 chktex-file 13 chktex-file 15 chktex-file 17 chktex-file 18 chktex-file 23 chktex-file 26 chktex-file 31 chktex-file 36 chktex-file 44
\documentclass[10pt]{article}
\input{/Users/mr.hassium/Desktop/Github/Hello-World/hassium.tex}
\def\htitle{Elementary Differential Geometry}
\def\hauthor{Hassium}\let\hfauthor\hauthor
\begin{document}
\hstart
\section{Calculus on Euclidean Space}
% 1.1 Euclidean Space
\begin{definition}
    The \hdef{Euclidean 3-space}, denoted ${\R}^{3}$, is the set of ordered triples of the form $p=({p}_{1},{p}_{2},{p}_{3})$, where ${p}_{i}\in\R$. An element of ${\R}^{3}$ is called a \hdef{point}.
\end{definition}
\par
Let $p=({p}_{1},{p}_{2},{p}_{3}),q=({q}_{1},{q}_{2},{q}_{3})\in{\R}^{3}$ and let $a\in\R$. Define the addition to be $p+q=({p}_{i}+{q}_{i})$ and define the scalar multiplication to be $ap=(a{p}_{i})$. The additive identity $0=(0,0,0)$ is called the \hdef{origin} of ${\R}^{3}$. It is trivial that ${\R}^{3}$ is a vector space over $\R$.
\begin{definition}
    Let $x$, $y$, and $z$ be real-valued functions on ${\R}^{3}$ such that for all $p=({p}_{1},{p}_{2},{p}_{3})\in{\R}^{3}$, $x(p)={p}_{1}$. $y(p)={p}_{2}$, and $z(p)={p}_{3}$. We call $x$, $y$, and $z$ the \hdef{natural coordinate functions} of ${\R}^{3}$.
\end{definition}
\par
Let $x$, $y$, and $z$ be the natural coordinate functions, rewrite $x={x}_{1}$, $y={x}_{2}$, and $z={x}_{3}$. Then we have $p=({p}_{i})=({x}_{i}(p))$.
\begin{definition}
    A real-valued function $f$ on ${\R}^{3}$ is \hdef{differentiable} if all partial derivatives exist and continuous.
\end{definition}
\par
Let $p=({p}_{1},{p}_{2},{p}_{3}),q=({q}_{1},{q}_{2},{q}_{3})\in{\R}^{3}$. Recall that the dot product is defined to be $p\cdot q=\sum{p}_{i}{q}_{i}$ and the norm is defined to be $\norm{p}=\sqrt{p\cdot p}=\sqrt{\sum{p}_{i}^{2}}$. 
\begin{definition}
    A subset $O\sub{\R}^{3}$ is \hdef{open} if for all $p\in O$, there exists $\ve>0$ such that $\{x\in{\R}^{3}\mid\norm{x-p}<\ve\}\sub O$. 
\end{definition}
\par
Let $f:O\to\R$ be a function defined on an open set. The differetiability of $f$ at $p$ can be determined entirely from values of $f$ on $O$. This means that differetiation is a local operation. We will discuss this later.
% 1.2 Tangent Vectors
\begin{definition}
    A \hdef{tangent vector} ${v}_{p}$ is an ordered pair ${v}_{p}=(v,p)$, where $v,p\in{\R}^{3}$. Here $v$ is called the \hdef{vector part} and $p$ is called its \hdef{point of application}. Two tangent vectors are said to be \hdef{parallel} if they have the same vector part and different points of application.
\end{definition}
\begin{definition}
    Let $p\in{\R}^{3}$. The \hdef{tangent space} at $p$, denoted ${T}_{p}({\R}^{3})$, is the set of all tangent vectors that have $p$ as point of application.
\end{definition}
\par
Fix a tangent space ${T}_{p}({\R}^{3})$ and let ${T}_{p}({\R}^{3})$ adapt the operations from ${\R}^{3}\times{\R}^{3}$. We have a natural linear map $f:{T}_{p}({\R}^{3})\to{\R}^{3}$ defined by ${v}_{p}\to v$ and it is trivially an isomorphism.
\begin{definition}
    A \hdef{vector field} $V$ on ${\R}^{3}$ is a function $V:{\R}^{3}\to{\Du}_{p\in{\R}^{3}}{T}_{p}({\R}^{3})$ such that for all $p\in{\R}^{3}$, $V(p)\sub{T}_{p}({\R}^{3})$.
\end{definition}
\par
Let $V$ and $W$ be vector field. Let $f$ be a real-valued function. For all $p\in{\R}^{3}$, define $V+W$ by $(V+W)(p)=V(p)+W(p)$ and $(fV)(p)=f(p)V(p)$.
\begin{definition}
    Let ${U}_{1}$, ${U}_{2}$, and ${U}_{3}$ be vector fields on ${\R}^{3}$ such that ${U}_{1}(p)={(1,0,0)}_{p}$, ${U}_{2}(p)={(0,1,0)}_{p}$, and ${U}_{3}(p)={(0,0,1)}_{p}$ for all $p\in{\R}^{3}$. We call $({U}_{1},{U}_{2},{U}_{3})$ the \hdef{natural frame field} on ${\R}^{3}$.
\end{definition}
\begin{proposition}
    Let $V$ be a vector field on ${\R}^{3}$. There are three uniquely determined real-valued functions ${v}_{1}$, ${v}_{2}$, and ${v}_{3}$ on ${\R}^{3}$ such that $V={v}_{1}{U}_{1}+{v}_{2}{U}_{2}+{v}_{3}{U}_{3}$.
\end{proposition}
\begin{proof}
    For all $p\in{\R}^{3}$, $V(p)={({v}_{1}(p),{v}_{2}(p),{v}_{3}(p))}_{p}={v}_{1}(p){(1,0,0)}_{p}+{v}_{2}(p){(0,1,0)}_{p}+{v}_{3}(p){(0,0,1)}_{p}={v}_{1}(p){U}_{1}(p)+{v}_{2}(p){U}_{2}(p)+{v}_{3}{U}_{3}(p)$, hence $V=\sum{v}_{i}{U}_{i}$.
\end{proof}
\par
The functions ${v}_{1}$, ${v}_{2}$, and ${v}_{3}$ are called the \hdef{Euclidean coordinate functions} on $V$.
\begin{definition}
    A vector field $V$ is \hdef{differentiable} if its Euclidean coordinate functions are differentiable.
\end{definition}
% 1.3 Directional Derivatives
\begin{definition}
    Let $f$ be a differentiable real-valued function on ${\R}^{3}$ and let ${v}_{p}$ be a tangent vector on ${\R}^{3}$. The \hdef{directional derivative} of $f$ with respect to ${v}_{p}$, denoted ${v}_{p}[f]$, is defined to be $(\d/\d t)f(p+tv)$ at $t=0$.
\end{definition}
\begin{remark}
    We will not write the restriction every time for convenience.
\end{remark}
\begin{proposition}
    Let ${v}_{p}={({v}_{1},{v}_{2},{v}_{3})}_{p}$ be a tangent vector, then ${v}_{p}[f]=\sum{v}_{i}(\pa f/\pa{x}_{i})(p)$.
\end{proposition}
\begin{proof}
    Let $p=({p}_{1},{p}_{2},{p}_{3})$. Then ${v}_{p}[f]=(\d/\d t)f(p+tv){\vert}_{t=0}=\sum(\pa f/\pa z)(p)\cdot(\d/\d t)({p}_{i}+t{v}_{i})=\sum(\pa f/\pa{x}_{i})(p){v}_{i}$.
\end{proof}
\begin{example}
    Consider $f={x}^{2}yz$ with $p=(1,1,0)$ and $v=(1,0,-3)$. By the definition, $p+tv=(1+t,1,-3t)$, so ${v}_{p}[f]=(\d/\d t)(-3{t}^{3}-6{t}^{2}-3t)=-3$. Since $(\pa f/\pa x)=2xyz$, $(\pa f/\pa y)={x}^{2}z$, and $(\pa f/\pa z)={x}^{2}y$, we have $(\pa f/\pa x)(p)=(\pa f/\pa y)(p)=0$ and $(\pa f/\pa z)(p)=1$, so ${v}_{p}[f]=-3$.
\end{example}
\begin{proposition}
    Let $f$ and $g$ be functions on ${\R}^{3}$. Let ${v}_{p}$ and ${w}_{p}$ be tangent vectors. For all $a,b\in\R$, the following properties hold.
    \begin{enumerate}
        \item $(a{v}_{p}+b{w}_{p})[f]=a{v}_{p}[f]+b{w}_{p}[f]$.
        \item ${v}_{p}[af+bg]=a{v}_{p}[f]+b{v}_{p}[g]$.
        \item ${v}_{p}[fg]={v}_{p}[f]g(p)+f(p){v}_{p}[g]$.
    \end{enumerate}
\end{proposition}
\begin{proof}
    (\rom1) We have $(a{v}_{p}+b{w}_{p})[f]=\sum(a{v}_{i}+b{w}_{i})(\pa f/\pa{x}_{i})(p)=\sum a{v}_{i}(\pa f/\pa{x}_{i})+\sum b{w}_{i}(\pa f/\pa{x}_{i})(p)=a{v}_{p}[f]+b{w}_{p}[f]$. (\rom2) We have ${v}_{p}[af+bg]=\sum{v}_{i}(\pa(af+bg)/\pa{x}_{i})(p)=\sum{v}_{i}(\pa(af)/\pa{x}_{i})(p)+\sum{v}_{i}(\pa(bg)/\pa{x}_{i})(p)=a{v}_{p}[f]+b{v}_{p}[g]$. (\rom3) We have ${v}_{p}[fg]=\sum{v}_{i}(\pa(fg)/\pa{x}_{i})(p)=\sum{v}_{i}(\pa f/\pa{x}_{i})(p)g(p)+f(p)\sum{v}_{i}(\pa g/\pa{x}_{i})(p)={v}_{p}[f]g(p)+f(p){v}_{p}[g]$.
\end{proof}
\par
Let $V$ be a vector field, we define $V[f]$ at $p\in{\R}^{3}$ to be $V(p)[f]$. By the convention, ${U}_{i}(p)[f]=(\pa f/\pa{x}_{i})(p)$.
\begin{proposition}
    Let $V$ and $W$ be vector fields. Let $f$, $g$, and $h$ be real-valued functions. For all $a,b\in\R$, the following properties hold.
    \begin{enumerate}
        \item $(fV+gW)[h]=fV[h]+gW[h]$.
        \item $V[af+bg]=aV[f]+bV[g]$.
        \item $V[fg]=V[f]g+fV[g]$.
    \end{enumerate}
\end{proposition}
\begin{proof}
    (\rom1) For all $p\in{\R}^{3}$, $(fV+gW)(p)[h]=(f(p)V(p)+g(p)W(p))[h]=fV[h]+gW[h]$. (\rom2) For all $p\in{\R}^{3}$, $V(p)[af+bg]=aV(p)[f]+bV(p)[g]$. (\rom3) For all $p\in{\R}^{3}$, $V(p)[f]g(p)+f(p)V(p)[g]=V[f](p)g(p)+f(p)V[g](p)=(V[f]g+fV[g])(p)$.
\end{proof}
\begin{example}
    Let $V=x{U}_{1}-{y}^{2}{U}_{3}$ and let $f={x}^{2}y+{z}^{3}$. Then $V[f]=x{U}_{1}[{x}^{2}y]+x{U}_{1}[{z}^{3}]-{y}^{2}{U}_{3}[{x}^{2}y]-{y}^{2}{U}_{3}[{z}^{3}]=2{x}^{2}y-3{y}^{2}{z}^{2}$.
\end{example}
% 1.4 Curves in {\R}^{3}
\par
Let $I\sub\R$ be an open interval. Let $\al:I\to{\R}^{3}$ be a function. We can rewrite $\al(t)$ as $({\al}_{1}(t),{\al}_{2}(t),{\al}_{3}(t))$, where ${\al}_{i}:I\to\R$. We say $\al$ is \hdef{differentiable} if ${\al}_{i}$ are differentiable.
\begin{definition}
    A \hdef{curve} in ${\R}^{3}$ is a differentiable function $\al:I\to{\R}^{3}$, where $I\sub\R$ is an open interval.
\end{definition}
\begin{example}
    A curve $\al:\R\to{\R}^{3}$ defined by $\al(t)=p+tq$, where $\al(0)=p$ and $q\ne 0$, is called a \hdef{straight line}.
\end{example}
\begin{example}
    Here are some examples of curves.
    \begin{enumerate}
        \item The cruve $\al:\R\to{\R}^{3}$ defined by $\al(t)=(a\cos t,a\sin t,bt)$.
        \item The cruve $\al:\R\to{\R}^{3}$ defined by $\al(t)=(1+\cos t,\sin t,2\sin(t/2))$.
        \item The cruve $\al:\R\to{\R}^{3}$ defined by $\al(t)=({e}^{t},{e}^{-t},\sqrt{2}t)$.
        \item The cruve $\al:\R\to{\R}^{3}$ defined by $\al(t)=(3t-{t}^{3},3{t}^{2},3t+{t}^{3})$.
    \end{enumerate}
\end{example}
\begin{definition}
    Let $\al:I\to{\R}^{3}$ be a curve with $\al=({\al}_{1},{\al}_{2},{\al}_{3})$. For all $t\in I$, the \hdef{velocity vector} of $\al$ at $t$ is the tangent vector $\al'(t)={((\d{\al}_{1}/\d t)(t),(\d{\al}_{2}/\d t)(t),(\d{\al}_{3}/\d t)(t))}_{\al(t)}$ at the point $\al(t)\in{\R}^{3}$. The curve $\al$ is said to be \hdef{regular} if ${\al}_{i}\ne 0$ for all $i$.
\end{definition}
\par
Consider the velocity vector $\al'(t)$, we can rewrite it by the natural frame fields, so $\al'(t)=\sum(\d{\al}_{i}/\d t)(t){U}_{i}(\al(t))$.
\begin{definition}
    Let $\al:I\to{\R}^{3}$ be a curve and let $h:J\to I$ be differentiable, where $J$ is an open interval of $\R$. The \hdef{reparametrization} of $\al$ by $h$ is the composition $\al\comp h:J\to{\R}^{3}$.
\end{definition}
\par
The composition of differentiable functions is differentiable, so any reparametrization is differentiable, which means it is a curve.
\begin{proposition}
    Let $\be$ be the reparametrization of $\al$ by $h$, then $\be'(s)=(\d h/\d s)(s)\al'(h(s))$.
\end{proposition}
\begin{proof}
    Rewrite $\be(s)=\al(h(s))$, then we have $\be'(s)={(\d({\al}_{i}{h}_{i})/\d s)(s)}_{\al(h(s))}={(\d{\al}_{i}/\d s)(h(s))\cdot(\d h/\d s)(s)}_{\al(h(s))}=(\d h/\d s)(s)\al'(h(s))$.
\end{proof}
\begin{proposition}
    Let $\al$ be a curve and let $f$ be a differentiable function on ${\R}^{3}$, then $\al'(t)[f]=(\d(f\al)/\d t)(t)$.
\end{proposition}
\begin{proof}
    We have $\al'(t)[f]=\sum(\d{\al}_{i}/\d t)(t)\cdot(\pa f/\pa{x}_{i})(\al(t))=(\d(f\al)/\d t)(t)$ by the chain rule.
\end{proof}
\par
Now we show a general idea of parametrizations. The proofs will be included in other sections when we have enough tools. Assume every result is correct for now.
% 1.5 1-Forms
\begin{definition}
    A \hdef{1-form} $\vp$ on ${\R}^{3}$ is a function $\vp:{\Cp}_{p\in{\R}^{3}}{T}_{p}({\R}^{3})\to\R$ such that for all $a,b\in\R$ and $v,w\in{T}_{p}({\R}^{3})$ for some $p\in{\R}^{3}$, $\vp(av+bw)=a\vp(v)+b\vp(w)$.
\end{definition}
\par
Given a 1-form $\vp$, for any point $p$, denote the restriction $\vp{\vert}_{{T}_{p}({\R}^{3})}:{T}_{p}({\R}^{3})\to\R$ by ${\vp}_{p}$, then ${\vp}_{p}$ is linear. Let $\vp$ and $\psi$ be 1-forms. Define the addition and scalar multiplication by $(\vp+\psi)(v)=\vp(v)+\psi(v)$ and $(f\vp)({v}_{p})=f(p)\vp({v}_{p})$. Given any 1-form $\vp$ and point $p$, ${\vp}_{p}$ is a linear functional in ${T}_{p}^{*}({\R}^{3})$, the dual space of ${T}_{p}({\R}^{3})$.
\begin{definition}
    Let $\vp$ be a 1-form and let $V$ be a vector field. For all $p\in{\R}^{3}$, define $\vp(V)(p)={\vp}_{p}(V(p))$. We say $\vp$ is \hdef{differentiable} if for every differentiable vector field $V$, the function $\vp(V)$ is differentiable.
\end{definition}
\par
Now let $V$ and $W$ be vector fields, we have $\vp(fV+gW)(p)=\vp((fV+gW)(p))=\vp(fV(p)+gW(p))=(f\vp(V)+g\vp(W))(p)$. Similarly, $(f\vp+g\psi)(V)=f\vp(V)+g\psi(V)$.
\begin{definition}
    If $f:{\R}^{3}\to\R$ is differentiable. The \hdef{differential} of $f$, denoted $\d f$, is the function $\d f({v}_{p})={v}_{p}[f]$ for all tangent vectors ${v}_{p}$.
\end{definition}
\par
Let ${v}_{p},{w}_{p}\in{T}_{p}({\R}^{3})$ and let $a,b\in\R$, then $\d f(a{v}_{p}+b{w}_{p})=(a{v}_{p}+b{w}_{p})[f]=a{v}_{p}[f]+b{w}_{p}[f]=a\d f({v}_{p})+b\d f({w}_{p})$. Hence $\d f$ is a 1-form.
\begin{example}
    Consider the natural coordinate functions ${x}_{i}$. We have $\d{x}_{i}({v}_{p})={v}_{p}[{x}_{i}]=\sum{v}_{i}(\pa{x}_{i}/\pa{x}_{j})(p)={v}_{i}$.
\end{example}
\begin{proposition}
    If $\vp$ is a 1-form on ${\R}^{3}$, then $\vp=\sum{f}_{i}\d{x}_{i}$, where ${f}_{i}=\vp({U}_{i})$.
\end{proposition}
\begin{proof}
    Let ${v}_{p}\in{T}_{p}({\R}^{3})$, then $\vp({v}_{p})=\vp(\sum{v}_{i}{U}_{i}(p))=\sum{v}_{i}\vp({U}_{i}(p))=\sum{v}_{i}{f}_{i}(p)=\sum{f}_{i}(p)\d{x}_{i}({v}_{p})=(\sum{f}_{i}\d{x}_{i})({v}_{p})$, hence $\vp=\sum{f}_{i}\d{x}_{i}$.
\end{proof}
\par
The functions ${f}_{1}$, ${f}_{2}$, and ${f}_{3}$ are called the \hdef{Euclidean coordinate functions} of the 1-form $\vp$.
\begin{proposition}
    Let $f$ be a differentiable function on ${\R}^{3}$, then $\d f=\sum(\pa f/\pa{x}_{i})\d{x}_{i}$.
\end{proposition}
\begin{proof}
    Let ${v}_{p}\in{T}_{p}({\R}^{3})$, then $\d f({v}_{p})={v}_{p}[f]=\sum{v}_{i}(\pa f/\pa{x}_{i})(p)=\sum(\pa f/\pa{x}_{i})(p)\d{x}_{i}({v}_{p})=(\sum(\pa f/\pa{x}_{i})\d{x}_{i})({v}_{p})$, hence $\d f=\sum(\pa f/\pa{x}_{i})\d{x}_{i}$.
\end{proof}
\par
Let $f$ and $g$ be differentiable functions on ${\R}^{3}$, then $\d(f+g)=\sum(\pa(f+g)/\pa{x}_{i})\d{x}_{i}=\sum(\pa f/\pa{x}_{i})\d{x}_{i}+\sum(\pa g/\pa{x}_{i})\d{x}_{i}=\d f+\d g$. Now we denote the multiplication to be $fg$.
\begin{proposition}
    Let $f$ and $g$ be differentiable functions on ${\R}^{3}$, then $\d(fg)=g\d f+f\d g$.
\end{proposition}
\begin{proof}
    We have $\d(fg)=\sum(\pa(fg)/\pa{x}_{i})\d{x}_{i}=\sum((\pa f/\pa{x}_{i})g+(\pa g/\pa{x}_{i})f)\d{x}_{i}=g\d f+f\d g$.
\end{proof}
\begin{proposition}
    Let $f:{\R}^{3}\to\R$ and $h:\R\to\R$ be differentiable, then $\d(h(f))=(\d h(f)/\d x)\d f$.
\end{proposition}
\begin{proof}
    We have $\d(h(f))=\sum(\pa h(f)/\pa{x}_{i})\d{x}_{i}$, by the chain rule, $(\pa h(f)/\pa{x}_{i})\d{x}_{i}=(\d h(f)/\d f)(\pa x/\pa{x}_{i})$, so $\d(h(f))=(\d f(h)/\d f)\d f$.
\end{proof}
\begin{example}
    Consider the function $f=({x}^{2}-1)y+({y}^{2}+2)z$. We have $\d f=\d(({x}^{2}-1)y)+\d(({y}^{2}+2)z)=y\d({x}^{2}-1)+({x}^{2}+1)\d y+z\d({y}^{2}+2)+({y}^{2}+2)\d z=2xy\d x+({x}^{2}+2yz-1)\d y+({y}^{2}+2)\d z$. Since ${v}_{p}[f]=\d f({v}_{p})$, ${v}_{p}[f]=2{p}_{1}{p}_{2}{v}_{1}+({p}_{1}^{2}+2{p}_{2}{p}_{3}-1){v}_{2}+({p}_{2}^{2}+2){v}_{3}$.
\end{example}
% 1.6 Differential Forms
\begin{definition}
    Let $V$ be the vector space ${\R}^{3}$ and denote the space of all $p$-linear forms on $V$ by ${\LA}^{p}({V}^{*})$. Every element of ${\LA}^{p}$ is called a \hdef{p-form}. Define the \hdef{wedge product} to be a function $\wedge:{\LA}^{a}({V}^{*})\times{\LA}^{b}({V}^{*})\to{\LA}^{a+b}({V}^{*})$ such that for $\om\in{\LA}^{m}({V}^{*})$, $\eta\in{\LA}^{n}({V}^{*})$, and ${v}_{1},\dots,{v}_{m+n}\in V$, the following properties hold.
    \begin{enumerate}
        \item $(\om\wedge\eta)({v}_{1},\dots,{v}_{m+n})=({\sum}_{\si\in{\Sg}_{m+n}}\sgn(\si)\om({v}_{\si(1)},\dots,{v}_{\si(m)})\eta({v}_{\si(m+1)},\dots,{v}_{\si(m+n)}))/(m!n!)$.
        \item $\om\wedge\eta={(-1)}^{mn}\eta\wedge\om$.
    \end{enumerate}
\end{definition}
\par
Generally, a $p$-form is of the form $\sum f(x,y,z)\d{x}^{i}\wedge\cdots\d{y}^{j}\wedge\cdots\d{z}^{k}\wedge\cdots$. We have $\d{x}_{i}\wedge\d{x}_{j}=-\d{x}_{j}\wedge\d{x}_{i}$. If $i=j$, then $\d{x}_{i}\wedge\d{x}_{i}=-\d{x}_{i}\wedge\d{x}_{i}$, so $\d{x}_{i}\wedge\d{x}_{i}=0$. It is trivial that $\wedge$ is bilinear and associative, that is,
\begin{enumerate}
    \item for ${\om}_{1},{\om}_{2}\in{\LA}^{m}({V}^{*})$, $\eta\in{\LA}^{n}({V}^{*})$, and $a,b\in\R$,$(a{\om}_{1}+b{\om}_{2})\wedge\eta=a({\om}_{1}\wedge\eta)+b({\om}_{2}\wedge\eta)$ and $\eta\wedge(a{\om}_{1}+b{\om}_{2})=a(\eta\wedge{\om}_{1})+b(\eta\wedge{\om}_{2})$;
    \item for $\om\in{\LA}^{m}({V}^{*})$, $\eta\in{\LA}^{n}({V}^{*})$, and $\theta\in{\LA}^{l}({V}^{*})$, $\om\wedge(\eta\wedge\theta)=(\om\wedge\eta)\wedge\theta$.
\end{enumerate}
Now given a space of $p$-forms ${\LA}^{p}({V}^{*})$ with basis $\{{e}_{1},{e}_{2},{e}_{3}\}$, the basis of its dual space is denoted by $\{{e}^{1},{e}^{2},{e}^{3}\}$. The basis of ${\LA}^{k}({V}^{*})$ is of the form ${e}^{{i}_{1}}\wedge\cdots\wedge{e}^{{i}_{k}}$, where $1\le{i}_{1}\le\cdots\le{i}_{k}\le 3$. In this case, the dimension of ${\LA}^{p}({V}^{*})$ is $3!/(p!(3-p)!)$. If $p>4$, then $\dim({\LA}^{p}({V}^{*}))=0$, so there are no $p$-forms on ${\R}^{3}$ if $p\ge 4$.
\begin{example}
    Let $\vp=x\d x-y\d y$, $\psi=z\d x+x\d z$, $\theta=z\d y$, and $\eta=y\d x\wedge\d z+x\d y\wedge\d z$.
    \begin{enumerate}
        \item $\vp\wedge\psi=xz\d x\wedge\d x+{x}^{2}\d x\wedge\d z-yz\d y\wedge\d x-yx\d y\wedge\d z=yz\d x\wedge\d y+{x}^{2}\d x\wedge\d z-yx\d y\wedge\d z$
        \item $\theta\wedge(\vp\wedge\psi)=y{z}^{2}\d x\wedge(\d y\wedge\d y)+{x}^{2}z\d x\wedge\d z\wedge\d y-xyz\d y\wedge\d z\wedge\d y=-{x}^{2}z\d x\wedge\d y\wedge\d z$
        \item $\vp\wedge\eta=xy\d x\wedge\d x\wedge\d z+{x}^{2}\d x\wedge\d y\wedge\d z-{y}^{2}\d y\wedge\d x\wedge\d z-xy\d y\wedge\d y\wedge\d z=({x}^{2}+{y}^{2})\d x\wedge\d y\wedge\d z$
    \end{enumerate}
\end{example}
\begin{proposition}
    Let $\vp$ and $\psi$ be 1-forms, then $\vp\wedge\psi=-\psi\wedge\vp$.
\end{proposition}
\begin{proof}
    Rewrite $\vp=\sum{f}_{i}\d{x}_{i}$ and $\psi=\sum{g}_{i}\d{x}_{i}$, then $\vp\wedge\psi=\sum{f}_{i}{g}_{i}\d{x}_{i}\d{x}_{j}=\sum-{g}_{i}{f}_{i}\d{x}_{j}\d{x}_{i}=-\psi\wedge\vp$.
\end{proof}
\begin{definition}
    Let $\vp=\sum{f}_{i}\d{x}_{i}$ be a 1-form on ${\R}^{3}$. The \hdef{exterior derivative} of $\vp$ is the 2-form $\d\vp=\sum\d{f}_{i}\wedge\d{x}_{i}$. Let $\psi=\sum{f}_{i,j}\d{x}_{i}\wedge\d{x}_{j}$ be a 2-form. The \hdef{exterior derivative} of $\psi$ is the 3-form $\d\psi=\sum\d{f}_{i,j}\wedge\d{x}_{i}\wedge\d{x}_{j}$. 
\end{definition}
\par
Let $a,b\in\R$. Let $\vp=\sum{f}_{i}\d{x}_{i}$ and $\psi=\sum{g}_{i}\d{x}_{i}$ be 1-forms. Then $\d(a\vp+b\psi)=\d(\sum(a{f}_{i}+b{g}_{i})\d{x}_{i})=\sum\d(a{f}_{i}+b{g}_{i})\wedge\d{x}_{i}$, since the differential is linear, the exterior derivative is linear.
\begin{proposition}
    Let $f,g:{\R}^{3}\to\R$ be functions and let $\vp$ and $\psi$ be 1-forms. Then $\d(f\vp)=\d f\wedge\vp+f\d\vp$ and $\d(\vp\wedge\psi)=\d\vp\wedge\psi-\vp\wedge\d\psi$.
\end{proposition}
\begin{proof}
    (\rom1) Let $\vp=\sum{g}_{i}\d{x}_{i}$, then $f\vp=\sum f{g}_{i}\d{x}_{i}$, so $\d(f\vp)=\sum(f\d{g}_{i}+{g}_{i}\d f)\wedge\d{x}_{i}=\sum f\d{g}_{i}\wedge\d{x}_{i}+\sum{g}_{i}\d f\wedge\d{x}_{i}=f\d\vp+\d f\wedge\vp$. (\rom2) Since $\d{x}_{i}\wedge\d{x}_{i}=0$, without lose of generality, let $\vp=f\d x$ and let $\psi=g\d y$. Then $\d(\vp\wedge\psi)=\d(fg\d x\wedge\d y)=\d(fg)\wedge\d x\wedge\d y=(f\d g+g\d f)\wedge\d x\wedge\d y=f\d g\wedge\d x\wedge\d y+g\d f\wedge\d x\wedge\d y$. For the right hand side, $\d\vp\wedge\psi=\d f\wedge\d x\wedge g\d y=g\d f\wedge\d x\wedge\d y$ and $\vp\wedge\d\psi=f\d x\wedge\d g\wedge\d y=-f\d g\wedge\d x\wedge\d y$, hence $\d(\vp\wedge\psi)=\d\vp\wedge\psi-\vp\wedge\d\psi$.
\end{proof}
% 1.7 Mappings
\begin{definition}
    Let $F:{\R}^{n}\to{\R}^{m}$ and let ${f}_{1},\dots,{f}_{m}:{\R}^{n}\to\R$ such that $F(p)=({f}_{1}(p),\dots,{f}_{m}(p))$ for all $p\in{\R}^{n}$. The functions ${f}_{i}$ are called the \hdef{Euclidean coordinate functions} of $F$ and we denote $F=({f}_{1},\dots,{f}_{m})$.
\end{definition}
\begin{definition}
    Let $F:{\R}^{n}\to{\R}^{m}$ and $F=({f}_{1},\dots,{f}_{m})$, we say $F$ is \hdef{differentiable} if all ${f}_{i}$ are differentiable. If $F$ is differentiable, we say $F$ is a \hdef{mapping} from ${\R}^{n}$ to ${\R}^{m}$.
\end{definition}
\begin{definition}
    Let $\al:I\to{\R}^{n}$ be a curve and let $F:{\R}^{n}\to{\R}^{m}$ be a mapping. Then the composite function $\be=F(\al):I\to{\R}^{m}$ is a curve in ${\R}^{m}$ called the \hdef{image} of $\al$ under $F$.
\end{definition}
\par
To examine the effect of a mapping, it suffices to take a proper $\al$ and check the image of it.
\begin{example}
    The function $F:{\R}^{3}\to{\R}^{3}$ defined by $F=(x-y,x+y,2z)$ is a mapping. Trivially, $F$ is a linear map, so $F$ is determined by $F({u}_{i})$.
\end{example}
\begin{example}
    Consider the mapping $F:{\R}^{2}\to{\R}^{2}$ defined by $F=({u}^{2}-{v}^{2},2uv)$. Let $\al:I\to{\R}^{2}$ defined by $\al(t)=(r\cos t,r\sin t)$, where $0\le t\le 2\pi$. The image is $\be(t)=({r}^{2}\cos 2t,{r}^{2}\sin 2t)$. This curve takes two counterclockwise trips around the circle of radius ${r}^{2}$ centered at the origin. Therefore, $F$ wraps ${\R}^{2}$ around itself twice.
\end{example}
\begin{definition}
    Let $F:{\R}^{n}\to{\R}^{m}$ be a mapping and let ${v}_{p}\in{T}_{p}({\R}^{n})$. The \hdef{tangent map} of $F$, denoted ${F}_{*}({v}_{p})$, is defined to be $(\d/\d t)F(p+tv)$ at $t=0$.
\end{definition}
\par
Fix some mapping $F:{\R}^{n}\to{\R}^{m}$. For every $p\in{\R}^{n}$, it induces a tangent map of $F$ at $p$, denoted ${F}_{*p}$.
\begin{proposition}
    Let $F=({f}_{1},\dots,{f}_{m}):{\R}^{n}\to{\R}^{m}$ be a mapping. If ${v}_{p}\in{T}_{p}({\R}^{n})$, then ${F}_{*p}({v}_{p})={(v[{f}_{1}],\dots,v[{f}_{m}])}_{F(p)}$.
\end{proposition}
\begin{proof}
    Fix ${v}_{p}\in{T}_{p}({\R}^{n})$. We have ${F}_{*p}=(\d/\d t)F(p+tv){\vert}_{t=0}=(\d/\d t)({f}_{i}(p+tv)){\vert}_{t=0}={({v}_{p}[{f}_{1}],\dots,{v}_{p}[{f}_{m}])}_{F(p)}$.
\end{proof}
\begin{proposition}
    Let $F=({f}_{1},\dots,{f}_{m}):{\R}^{n}\to{\R}^{m}$ be a mapping. For all $p\in{T}_{p}({\R}^{n})$,  the tangent map ${F}_{*p}:{T}_{p}({\R}^{n})\to{T}_{F(p)}({\R}^{m})$ is a linear map.
\end{proposition}
\begin{proof}
    Fix $p\in{\R}^{n}$. Let $a,b\in\R$ and let ${v}_{p},{w}_{p}\in{T}_{p}({\R}^{n})$. We have ${F}_{*p}(a{v}_{p}+b{w}_{p})={((a{v}_{p}+b{w}_{p})[{f}_{i}])}_{F(p)}={(a{v}_{p}[{f}_{i}])}_{F(p)}+{(b{w}_{p}[{f}_{i}])}_{F(p)}=a{F}_{*p}({v}_{p})+b{F}_{*p}({w}_{p})$.
\end{proof}
\begin{proposition}
    Let $F:{\R}^{n}\to{\R}^{m}$ be a mapping and let $\be$ be the image of some curve $\al$ in ${\R}^{n}$, then $\be'={F}_{*}(\al')$.
\end{proposition}
\begin{proof}
    Let $F=({f}_{1},\dots,{f}_{m})$. We have ${F}_{*}(\al'(t))={(\al'(t)[{f}_{i}])}_{F(\al(t))}={(\d{f}_{i}(\al(t))/\d t)}_{F(\al(t))}=\be'(t)$.
\end{proof}
\par
Let $\{{U}_{j}\}$ and $\{\wb{{U}_{i}}\}$ be the natural frame fields of ${\R}^{n}$ and ${\R}^{m}$, respectively.
\begin{proposition}
    Let $F=({f}_{1},\dots,{f}_{m}):{\R}^{n}\to{\R}^{m}$ be a mapping. Then ${F}_{*}({U}_{j}(p))={\sum}_{i=1}^{m}(\pa{f}_{i}/\pa{x}_{j})(p)\wb{{U}_{i}}(F(p))$, where $1\le j\le n$.
\end{proposition}
\begin{proof}
    Recall that ${U}_{j}[{f}_{i}]=\pa{f}_{i}/\pa{x}_{j}$, so the proposition trivially holds.
\end{proof}
\begin{definition}
    Let $F=({f}_{1},\dots,{f}_{m}):{\R}^{n}\to{\R}^{m}$ be a mapping. The \hdef{Jacobian matrix} of $F$ at $x\in{\R}^{n}$ is the matrix
    \begin{center}
        ${J}_{F}(x)=\begin{pmatrix}
            \pa{f}_{1}/\pa{x}_{1}(x) & \cdots & \pa{f}_{1}/\pa{x}_{n}(x) \\
            \vdots & \ddots & \vdots \\
            \pa{f}_{m}/\pa{x}_{1}(x) & \cdots & \pa{f}_{m}/\pa{x}_{n}(x)
        \end{pmatrix}$.
    \end{center}
\end{definition}
\begin{definition}
    Let $F:{\R}^{n}\to{\R}^{m}$ be a mapping. We say $F$ is \hdef{regular} if for all $p\in{\R}^{n}$, ${F}_{*p}$ is injective.
\end{definition}
\par
Notice that ${J}_{F}(p)\cdot v={F}_{*p}$, so ${J}_{F}(p)$ is the matrix representation of ${F}_{*p}$.
\begin{definition}
    A mapping is a \hdef{diffeomorphism} if it has a differentiable inverse mapping.
\end{definition}
\begin{definition}
    A \hdef{topological space} $(X,\T)$ consists of two sets $X$ and $\T$, where $\T\sub\ps(X)$, that satisfies the following properties.
    \begin{enumerate}
        \item $\es,X\in\T$.
        \item Any union of elements in $\T$ is also in $\T$.
        \item Any finite intersection of elements in $\T$ is also in $\T$.
    \end{enumerate}
    The collection $\T$ is called a \hdef{topology} on $X$.
\end{definition}
\begin{definition}
    Let $(X,\T)$ be a topological space. A subset $U\sub X$ is said to be \hdef{open} if $U\in\T$. Let $x\in X$, a \hdef{neighborhood} of $x$ is an open set ${U}_{x}$ that contains $x$.
\end{definition}
\par
Let $U\sub\R$. We say $U$ is open in the standard topology $\T$ on $\R$ if for every $x\in U$, there exists $\ve>0$ such that $(x-\ve,x+\ve)\sub U$. Trivially, $\es,\R\in\T$. Let ${\{{U}_{i}\}}_{i\in I}$ be open sets, then for each ${U}_{i}$ and $x\in{U}_{i}$, there exists a corresponding ${\ve}_{i,x}$. For any $x\in{\Un}_{i\in I}{U}_{i}$, $x\in{U}_{i}$ for some $i\in I$. Pick $\ve={\ve}_{i,x}$, then $(x-\ve,x+\ve)\sub {U}_{i}\sub{\Un}_{i\in I}{U}_{i}$. For any $x\in{\In}_{i=1}^{n}{U}_{i}$, pick $\ve=\min\{{\ve}_{i,x}\}$, then $(x-\ve,x+\ve)\sub{U}_{i}$ for $1\le i\le n$, so $(x-\ve,x+\ve)\sub{\In}_{i=1}^{n}{U}_{i}$. The standard topology on $\R$ is indeed a topology.
\begin{definition}
    Let $(X,{\T}_{X})$ and $(Y,{\T}_{Y})$ be topological spaces. A subset $W\sub X\times Y$ is open in the \hdef{product topology} on $X\times Y$ if for all $(x,y)\in W$, there exist neighborhoods ${U}_{x}\in{\T}_{X}$ and ${V}_{y}\in{\T}_{Y}$ such that ${U}_{x}\times{V}_{y}\sub W$.
\end{definition}
\par
Denote the product topology by $\T$. We have $\es\in\T$ vacuously. For all $(x,y)\in X\times Y$, ${U}_{x}\sub X$, and ${V}_{Y}\sub Y$, ${U}_{X}\times{V}_{Y}\sub X\times Y$, so $X\times Y\in\T$. Let ${\{{W}_{i}\}}_{i\in I}$ be open sets. For all $(x,y)\in{\Un}_{i\in I}{W}_{i}$, there exist ${W}_{i}$ and ${W}_{j}$ such that $x\in{W}_{i}$ and $y\in{W}_{j}$. Pick the corresponding neighborhood in each set, then ${U}_{x}\times{V}_{y}\sub{W}_{i}\cup{W}_{j}\sub{\Un}_{i\in I}{W}_{i}$. For all $(x,y)\in{\In}_{i=1}^{n}{W}_{i}$, $(x,y)\in{W}_{i}$. For each ${W}_{i}$, we have a corresponding pair $({U}_{i,x},{V}_{i,y})$. Now consider $U={\In}_{i=1}^{n}{U}_{i,x}\in{\T}_{X}$ and $V={\In}_{i=1}^{n}{V}_{i,y}\in{\T}_{Y}$, we have $U\times V\sub{W}_{i}$, so $U\times V\sub{\In}_{i=1}^{n}{W}_{i}$. The standard topology on ${\R}^{n}$ is the product topology of $n$ copies of the standard topology on $\R$.
\begin{theorem}[inverse function theorem]
    Let $F:{\R}^{n}\to{\R}^{n}$ be a mapping. If ${F}_{*p}$ is injective at some $p\in{\R}^{n}$, then there exists a neighborhood $U$ of $p$ such that $F{\vert}_{U}:U\to V$, where $V$ is open, is a diffeomorphism.
\end{theorem}
\par
We will discuss more on the proof of this theorem and its application later.
\section{Frame Fields}
% 2.1 Dot Product
\begin{definition}
    Let $p,q\in{\R}^{3}$. The \hdef{Euclidean distance} from $p$ to $q$ is the number $d(p,q)=\norm{p-q}$.
\end{definition}
\begin{definition}
    Let ${v}_{p},{w}_{p}\in{T}_{p}({\R}^{3})$ be tangent vectors. The \hdef{dot product} of ${v}_{p}$ and ${w}_{p}$ is defined to be ${v}_{p}\cdot{w}_{p}=v\cdot w$.
\end{definition}
\par
Equivalently, the norm on every tangent space ${T}_{p}({\R}^{3})$ is the composition of the canonical isomorphism ${T}_{p}({\R}^{3})\to{\R}^{3}$ with the norm on ${\R}^{3}$.
\begin{definition}
    A set of three pairwise orthogonal unit vectors tangent to ${\R}^{3}$ at $p$ is called a \hdef{frame} at $p$.
\end{definition}
\par
By the definition, $\{{e}_{1},{e}_{2},{e}_{3}\}$ is a frame at $p$ if and only if ${e}_{i}\in{T}_{p}({\R}^{3})$ and ${e}_{i}\cdot{e}_{j}={\delta}_{i,j}$.
\begin{proposition}
    Let $\{{e}_{1},{e}_{2},{e}_{3}\}$ be a frame at $p\in{\R}^{3}$. If ${v}_{p}\in{T}_{p}({\R}^{3})$, then ${v}_{p}=\sum(v\cdot{e}_{i}){e}_{i}$.
\end{proposition}
\begin{proof}
    Let ${c}_{1},{c}_{2},{c}_{3}\in\R$ such that $\sum{c}_{i}{e}_{i}=0$. For all $1\le j\le 3$, $0=(\sum{c}_{i}{e}_{i})\cdot{e}_{j}=\sum{c}_{i}({e}_{i}\cdot{e}_{j})={c}_{j}$, so $\{{e}_{1},{e}_{2},{e}_{3}\}$ is a basis of ${T}_{p}({\R}^{3})$. Rewrite ${v}_{p}=\sum{a}_{i}{e}_{i}$. For all $1\le j\le 3$, ${v}_{p}\cdot{e}_{j}=\sum{a}_{i}{e}_{i}\cdot{e}_{j}={a}_{j}$. Hence ${v}_{p}=\sum({v}_{p}\cdot{e}_{i}){e}_{i}$.
\end{proof}
\par
For any frame $\{{e}_{1},{e}_{2},{e}_{3}\}$ at $p$ and $a,b\in{T}_{p}({\R}^{3})$, if $a=\sum{a}_{i}{e}_{i}$ and $b=\sum{b}_{i}{e}_{i}$, we always have $a\cdot b=\sum{a}_{i}{b}_{i}$.
\begin{definition}
    Let $\{{e}_{1},{e}_{2},{e}_{3}\}$ be a frame at $p\in{\R}^{3}$ with ${e}_{i}={({a}_{i,1},{a}_{i,2},{a}_{i,3})}_{p}$, then the \hdef{attitude matrix} of the frame is defined to be the matrix
    \begin{center}
        $A=\begin{pmatrix}
            {a}_{1,1} & {a}_{1,2} & {a}_{1,3} \\
            {a}_{2,1} & {a}_{2,2} & {a}_{2,3} \\
            {a}_{3,1} & {a}_{3,2} & {a}_{3,3}
        \end{pmatrix}$.
    \end{center}
\end{definition}
\par
Consider the transpose $\tp{A}$ of $A$, for each column of $\tp{A}A$, we have ${e}_{i}{e}_{i}=1$, so $\tp{A}A=I$ and $A$ is orthogonal.
\begin{definition}
    Let ${v}_{p}={({v}_{1},{v}_{2},{v}_{3})}_{p},{w}_{p}={({w}_{1},{w}_{2},{w}_{3})}_{p}\in{T}_{p}({\R}^{3})$ for some $p\in{\R}^{3}$. The \hdef{cross product} of ${v}_{p}$ and ${w}_{p}$, denoted ${v}_{p}\times{w}_{p}$, is the tangent vector
    \begin{center}
        ${v}_{p}\times{w}_{p}=\begin{vmatrix}
            {U}_{1}(p) & {U}_{2}(p) & {U}_{3}(p) \\
            {v}_{1} & {v}_{2} & {v}_{3} \\
            {w}_{1} & {w}_{2} & {w}_{3} 
        \end{vmatrix}$.
    \end{center}
\end{definition}
\begin{example}
    Let ${v}_{p}={(1,0,-1)}_{p}$ and let ${w}_{p}={(2,2,-7)}_{p}$, then ${v}_{p}\times{w}_{p}=2{U}_{1}(p)+5{U}_{2}(p)+2{U}_{3}(p)={(2,5,2)}_{p}$.
\end{example}
\par
It is trivial that $\times$ is linear and ${v}_{p}\times{w}_{p}=-{w}_{p}\times{v}_{p}$.
\begin{proposition}
    Let ${v}_{p},{w}_{p}\in{T}_{p}({\R}^{3})$ for some $p\in{\R}^{3}$. Then ${v}_{p}\times{w}_{p}$ is orthogonal to both ${v}_{p}$ and ${w}_{p}$. Moreover, ${\norm{{v}_{p}\times{w}_{p}}}^{2}=({v}_{p}\cdot{v}_{p})({w}_{p}\cdot{w}_{p})-{({v}_{p}\cdot{w}_{p})}^{2}$.
\end{proposition}
\begin{proof}
    Let ${v}_{p}={({v}_{1},{v}_{2},{v}_{3})}_{p},{w}_{p}={({w}_{1},{w}_{2},{w}_{3})}_{p}$. Then $({v}_{p}\times{w}_{p})\cdot{v}_{p}={v}_{1}({v}_{2}{w}_{3}-{v}_{2}{w}_{2})+{v}_{2}({v}_{3}{w}_{1}-{v}_{1}{w}_{3})+{v}_{3}({v}_{1}{w}_{2}-{v}_{2}{w}_{1})=0$. Similarly, $({v}_{p}\times{w}_{p})\cdot{w}_{p}=0$. We have $({v}_{p}\cdot{v}_{p})({w}_{p}\cdot{w}_{p})-{({v}_{p}\cdot{w}_{p})}^{2}=(\sum{v}_{i}^{2})(\sum{w}_{i}^{2})-{(\sum{v}_{i}{w}_{i})}^{2}=\sum{v}_{i}^{2}{w}_{j}^{2}-\sum{v}_{i}^{2}{w}_{i}^{2}-2{\sum}_{i<j}{v}_{i}{w}_{i}{v}_{j}-{w}_{j}={({v}_{2}{w}_{3}-{v}_{2}{w}_{2})}^{2}+{({v}_{3}{w}_{1}-{v}_{1}{w}_{3})}^{2}+{({v}_{1}{w}_{2}-{v}_{2}{w}_{1})}^{2}={\norm{{v}_{p}\times{w}_{p}}}^{2}$.
\end{proof}
% 2.2 Curves
\begin{definition}
    Let $\al:I\to{\R}^{3}$ be a curve. The \hdef{speed} of $\al$ at $t$ is the tangent vector $\norm{\al'(t)}$. The \hdef{arc length} of $\al$ from $t=a$ to $t=b$ is defined to be ${\int}_{a}^{b}\norm{\al'(t)}\d t$.
\end{definition}
\begin{proposition}
    Let $\al:I\to{\R}^{3}$ be a regular curve, then there exists a reparametrization $\be$ of $\al$ such that $\norm{\be'}=1$.
\end{proposition}
\begin{proof}
    Fix some $\al\in\R$ and consider the function $s:\R\to\R$ defined by $s(t)={\int}_{a}^{t}\norm{\al'(x)}\d x$. Since $\al$ is regular, $\norm{\al'(x)}>0$ for all $x$. By the inverse function theorem, $s(t)$ has an inverse $t(s)$. Define $\be(s)=\al(t(s))$, then $\norm{\be'}=\norm{(\d t/\d s)(s)\al'(t(s))}=(\d t/\d s)(s)\norm{\al'(t(s))}=(\d t/\d s)(s)\cdot(\d s/\d t)(t(s))=1$.
\end{proof}
\par
Such reparametrization $\be$ of $\al$ is called the \hdef{arc-length reparametrization} of $\al$.
\begin{example}
    Consider the curve $\al:I\to{\R}^{3}$ defined by $\al(t)=(a\cos t,a\sin t,bt)$ for some $a,b\in\R$. We have $\norm{\al'}=\sqrt{{a}^{2}{\sin}^{2}t+{a}^{2}{\sin}^{2}t+{b}^{2}}=c$, where ${c}^{2}={a}^{2}+{b}^{2}$. Now measure the arc length from $t=0$, then $s(t)={\int}_{0}^{c}c\d u=ct$, so $t(s)=s/c$, the arc-length reparametrization is therefore $\be(s)=\al(t(s))=(a\cos(s/c),a\sin(s/c),bs/c)$.
\end{example}
\begin{definition}
    A \hdef{vector field} $Y$ on a curve $\al:I\to{\R}^{3}$ is a function $Y:I\to{\Du}_{p\in\ran(\al)}{T}_{p}({\R}^{3})$ such that for all $t\in I$, $Y(t)\in{T}_{\al(t)}({\R}^{3})$.
\end{definition}
\par
Fix $t\in I$, then we can rewrite $Y(t)$ as $\sum{y}_{i}(t){U}_{i}(\al(t))$. The functions ${y}_{1}$, ${y}_{2}$, and ${y}_{3}$ are called the \hdef{Euclidean coordinate functions} on $Y$. 
\par
We define the addition, scalar multiplication, dot multiplication, and cross product on vector fields pointwisely. For a vector $Y=\sum{y}_{i}{U}_{i}$ on $\al$, the derivative of $Y$ is defined to be $Y'=\sum(\d{y}_{i}/\d t){U}_{i}$. Let $Y$ and $Z$ be vector fields on a curve $\al$. Fix $t$, rewrite $Y={({y}_{1},{y}_{2},{y}_{3})}_{\al(t)}$ and $Z={({z}_{1},{z}_{2},{z}_{3})}_{\al(t)}$. Consider $Y$ as a function ${Y}_{t}:I\to{T}_{\al(t)}({\R}^{3})\iso{\R}^{3}$, then for $a,b\in\R$ and a differentiable $f:\R\to\R$, $(aY+bZ)'=aY'+bZ'$ and $(fY)'=(\d f/\d t)Y+Y'f$. We also have $(Y\cdot Z)'=(\sum{y}_{i}{z}_{i})'=\sum{y}_{i}{z}_{i}'+\sum{y}_{i}'{z}_{i}=Y\cdot Z'+Y'\cdot Z$.
\begin{definition}
    Let $Y$ be a vector field on a curve $\al:I\to{\R}^{3}$. We say $Y$ is \hdef{parallel} if for all $t\in I$, $Y(t)$ have the same vector part.
\end{definition}
\begin{proposition}
    A curve $\al$ is constant if and only if $\al'=0$. A nonconstant curve $\al$ is a straight line if and only if $\al''=0$. A vector field $Y$ on $\al$ is parallel if and only if $Y'=0$.
\end{proposition}
\begin{proof}
    Rewrite $\al:I\to{\R}^{3}$ as $\al=({\al}_{i})$. (\rom1) The velocity $\al'=({\al}_{i}')$, then $\al'=0$ if and only if ${\al}_{i}'=0$. Hence $\al'=0$ if and only if ${\al}_{i}$ is a constant function. (\rom2) We have $\al''=({\al}_{i}'')$, so $\al''=0$ if and only if ${\al}_{i}={p}_{i}t+{q}_{i}$ for some ${p}_{i},{q}_{i}\in\R$. Hence $\al''=0$ if and only if $\al=pt+q$, where $p=({p}_{i})$ and $q=({q}_{1})$. (\rom3) Fix $t$ and let $Y={({y}_{i})}_{\al(t)}$, then $Y'=\sum{y}_{i}'{U}_{i}=0$, which means ${y}_{i}$ are constant functions. Hence $Y$ is parallel if and only if $Y'=0$.
\end{proof}
% 2.3 The Frenet Formulas
\begin{definition}
    Let $\be:I\to{\R}^{3}$ be a curve. Then we call $T=\be'$ the \hdef{unit tangent field} of $\be$. The function $\ka(s)=\norm{T'(s)}$ is called the \hdef{curvature} of $\be$. 
\end{definition}
\begin{remark}
    We shall only consider the cases where $\ka\ne 0$.
\end{remark}
\begin{definition}
    Let $\be:I\to{\R}^{3}$ be a curve. Then we call $N=T'/\ka$ the \hdef{principal normal vector field} of $\be$. The vector field $B=T\times N$ on $\be$ is called the \hdef{binormal vector field} of $\be$.
\end{definition}
\begin{proposition}
    Let $\be$ be a curve in ${\R}^{3}$ with $\ka>0$ and $\norm{\be'}=1$. Then the three vector fields $T$, $N$, and $B$ on $\be$ are unit vector fields that are mutually orthogonal at each point.
\end{proposition}
\begin{proof}
    Since $T=\be'$, $\norm{T}=\sqrt{T\cdot T}=1$, then $(T\cdot T)'=T\cdot T'+T'\cdot T=2T'\cdot T=0$, so $T\cdot T'=0$. For all $s\in\dom(\be)$, $\norm{N}=\norm{T'(s)}/\ka(s)=\norm{T'(s)}/\norm{T'(s)}=1$. Since $B=T\times N$, $B$ is orthogonal to $T$ and $N$. Moreover, ${\norm{B}}^{2}=\norm{T}\norm{N}-{(T\cdot N)}^{2}=1-0=1$.
\end{proof}
\begin{definition}
    Let $\be$ be a curve in ${\R}^{3}$ with $\ka>0$ and $\norm{\be'}=1$. Then $(T,N,B)$ is called the \hdef{Frenet frame field} of $\be$.
\end{definition}
\par
Now we claim that $B'$ can be written as a scalar multiple of $N$. Consider $B'=B'\cdot NN+B'\cdot TT+B'\cdot BB$. Differentiate $B\cdot T$, $B'\cdot T+T'\cdot B=0$, then $B'\cdot T=-B\cdot T'=-B\cdot\ka N=0$. Similarly, $B\cdot B'=0$, so $B'=B'\cdot NN$.
\begin{definition}
    Let $\be$ be a curve in ${\R}^{3}$ with $\ka>0$ and $\norm{\be'}=1$. Then the \hdef{torsion} of $\be$ is a function $\tau:I\to\R$ such that $B'=-\tau N$.
\end{definition}
\begin{theorem}[Frenet formulas]
    Let $\be:I\to{\R}^{3}$ be a curve with $\ka>0$ and $\norm{\be'}=1$. Then $T'=\ka N$, $N'=-\ka T+\tau B$, and $B'=-\tau N$.
\end{theorem}
\begin{proof}
    Rewrite $N'=N'\cdot TT+N'\cdot NN+N'\cdot BB$. Differentiate $T\cdot N$, we have $T'\cdot N+N'\cdot T=0$, so $N'\cdot T=-T'\cdot N=-(\ka N)\cdot N=-\ka$. Similarly, $N'\cdot B=-B'\cdot N=-(-\tau N)\cdot N=\tau$. Hence $N'=-\ka T+\tau B$.
\end{proof}
\begin{example}
    Consider the curve $\be:\R\to{\R}^{3}$ defined by $\be(s)=(a\cos(s/c),a\sin(s/c),bs/c)$, where $a>0$ and $c=\sqrt{{a}^{2}+{b}^{2}}$. It is trivial that $\norm{\be'}=1$. Here $T(s)=\be'(s)=(-a\sin(s/c)/c,a\cos(s/c)/c,b/c)$, then $T'(s)=(-a\cos(s/c)/{c}^{2},-a\sin(s/c)/{c}^{2},0)$, so $\ka(s)=\norm{T'(s)}=a/{c}^{2}>0$. We also have $N(s)=(-\cos(s/c),-\sin(s/c),0)$. Now $\be(s)=T(s)\times N(s)=(b\sin(s/c)/c,-b\cos(s/c)/c,a/c)$, then $B'(s)=(b/\cos(s/c)/{c}^{2},b\sin(s/c)/{c}^{2},0)$, so $\tau(s)=-B'(s)/N(s)=(-b/\cos(s/c)/{c}^{2},-b\sin(s/c)/{c}^{2},0)/(-\cos(s/c),-\sin(s/c),0)=b/{c}^{2}$.
\end{example}
\begin{definition}
    Let $p,q\in{\R}^{3}$ with $q\ne 0$. The \hdef{plane} through $p$ orthogonal to $q$ is the set $\{r\in{\R}^{3}\mid(r-p)\cdot q=0\}$. A curve $\be:I\to{\R}^{3}$ is said to be a \hdef{plane curve} if $\ran(\be)\sub P$, where $P$ is a plane in ${\R}^{3}$.
\end{definition}
\begin{proposition}
    Let $\be:I\to{\R}^{3}$ be a curve with $\norm{\be'}=1$ and $\ka>0$. Then $\be$ is a plane curve if and only if $\tau=0$. 
\end{proposition}
\begin{proof}
    $(\Ra)$ Let $\be$ be a plane curve, then there exists $p,q\in{\R}^{3}$ such that for all $s\in I$, $(\be(s)-p)\cdot q=0$. Consider $q$ as a constant vector field, so $((\be-p)\cdot q)'=(\be-p)'\cdot q+q'\cdot(\be-p)=\be'\cdot q-p'\cdot q=\be'\cdot q+q'\cdot\be=\be'\cdot q=\be''\cdot q=0$. Rewrite $q=q\cdot TT+q\cdot NN+q\cdot BB$, then $q=q\cdot BB$. We have $B=B\cdot BB=(B\cdot B)/(q\cdot B)q=(B\cdot B)/(q\cdot B)\norm{q}$. Since $\norm{B}=1$, $B=\pm q/\norm{q}$, which is a point, then $B'=0$, hence $\tau=0$. $(\La)$ Let $\tau=0$, then $B'=0$, so $B$ is constant. Define $f:I\to\R$ by $f(s)=(\be(s)-\be(0))\cdot B$, then $\d f/\d s=T(s)\cdot 0=0$ and $f(0)=0\cdot B=0$, so $f(s)=0$ for all $s$. Hence $\ran(\be)\sub\{r\mid(r-\be(0))\cdot B=0\}$.
\end{proof}
\begin{proposition}
    Let $\be$ be a curve in ${\R}^{3}$ with $\ka>0$, $\ka'=0$, $\norm{\be'}=1$, and $\tau=0$. Then $\be$ lies in a circle of radius $1/\ka$.
\end{proposition}
\begin{proof}
    Define $\ga:I\to{\R}^{3}$ by $\ga(s)=\be(s)+N(s)/\ka$, then $\ga'=T+N'/\ka$. By the Frenet formulas, $\ga'=T+(-\ka T+\tau B)/\ka=T-T=0$. Fix $t\in I$. For any $s\in I$, the distance $\norm{\be(s)-(\be(t)+N(t)/\ka)}=\norm{\be(s)-(\be(s)+N(s)/\ka)}=\norm{N(s)}/\ka=1/\ka$.
\end{proof}
% 2.4 Arbitraty-Speed Curves
\par
We have shown the properties of curves with unit speed. Now given a curve $\al:I\to{\R}^{3}$ with $\norm{\al'}\ne 1$, let $\wb{\al}$ be the arc-length reparametrization of $\al$, so $\norm{\wb{\al}'}=1$. Let $\wb{T}$, $\wb{k}$, $\wb{N}$, $\wb{B}$, and $\wb{\tau}$ be the corresponding functions of $\wb{\al}$. Define the $T$, $\la$, $N$, $B$, and $\tau$ of $\al$ to be those of $\wb{\al}$. We denote the speed of $\al$ by $v$.
\begin{theorem}[Frenet formulas]
    Let $\al$ be a regular curve on ${\R}^{3}$ with $\ka>0$, then $T'=\ka vN$, $N'=-\ka vT+\tau vB$, and $B'=-\tau vN$.
\end{theorem}
\begin{proof}
    Apply the Frenet formulas on $\wb{\al}$, then $\wb{T}'=\wb{\ka}\wb{N}$. Since $T'=\wb{T}'\d s/\d t=\wb{T}'v$, $T'=\wb{\ka}v\wb{N}$. Similarly, $N'=-\ka vT+\tau vB$ and $B'=-\tau vN$.
\end{proof}
\begin{proposition}
    Let $\al:I\to{\R}^{3}$ be regular. Then $\al'=vT$ and $\al''=(\d v/\d t)T+\ka{v}^{2}N$.
\end{proposition}
\begin{proof}
    We have $\al'=\wb{\al}'\d s/\d t=v\wb{T}=vT$, then $\al''=(\d v/\d t)T+T'v=(\d v/\d t)T+(\ka vN)\ka=(\d v/\d t)T+{v}^{2}\ka N$.
\end{proof}
\begin{proposition}
    Let $\al:I\to{\R}^{3}$ be a regular curve. Then $T=\al'/\norm{\al'}$, $N=B\times T$, $B=\al'\times\al''/\norm{\al'\times\al''}$, $\ka=\norm{\al'\times\al''}/{\norm{\al'}}^{3}$, and $\tau=(\al'\times\al'')\cdot\al'''/{\norm{\al'\times\al''}}^{2}$.
\end{proposition}
\begin{proof}
    Since $\norm{\al'}=v$, $\al'/\norm{\al'}=vT/v=T$. We have $\al'\times\al''=vT\times((\d v/\d t)T+\ka{v}^{2}N)=vT\times(\d v/\d t)T+vT\times\ka{v}^{2}N$, since $T\times T=0$, $\al'\times\al''=\ka{v}^{3}T\times N=\ka{v}^{3}B$. The norm $\norm{\al'\times\al''}=\norm{\ka{v}^{3}}=\ka{v}^{3}$, hence $B=\ka{v}^{3}B/(\ka{v}^{3})=\al'\times\al''/\norm{\al'\times\al''}$. Consider a lemma: for ${w}_{p},{v}_{p},{u}_{p}\in{T}_{p}({\R}^{3})$, $(u\times v)\times w=(u\cdot w)v-(v\cdot w)u$. (lemma) Rewrite $u=\sum{u}_{i}$, $v=\sum{v}_{i}$, and $w=\sum{w}_{i}$, then $(u\cdot w)v-(v\cdot w)u=(\sum{u}_{i}{w}_{j})\sum{v}_{i}-(\sum{v}_{i}{w}_{j})\sum{u}_{i}=u\times v\times w$. $\mathghost$ By the lemma, $B\times T=T\times N\times T=(T\cdot T)N-(N\cdot T)T=N-0=N$. We have shown $\norm{\al'\times\al''}=\ka{v}^{3}$, so $\norm{\al'\times\al''}/{\norm{\al'}}^{3}=\ka{v}^{3}/{v}^{3}=\ka$. Differentiate $\al''$, then $\al'''=(\d v/\d t)T'+({\d}^{2}v/\d{t}^{2})T+2\ka(\d v/\d t)vN+N'\ka{v}^{2}$. Since $B\cdot T=B\cdot N=0$, $\ka{v}^{3}B\cdot\al'''=\ka{v}^{3}B\cdot((\d v/\d t)T'+N'\ka{v}^{2})$, by the Frenet formulas, $T'$ term becomes 0 and $N'$ term becomes $\tau vB$. Now $(\al'\times\al'')\cdot\al'''={\ka}^{2}{v}^{6}\tau B\cdot B={\ka}^{2}{v}^{6}\tau$. Hence $(\al'\times\al'')\cdot\al'''/{\norm{\al'\times\al''}}^{2}={\ka}^{2}{v}^{6}\tau/({\ka}^{2}{v}^{6})=\tau$.
\end{proof}
\begin{example}
    Consider the curve $\al:\R\to{\R}^{3}$ defined by $\al(t)=(3t-{t}^{3},3{t}^{2},3t+{t}^{3})$, then $\al'=(3-3{t}^{2},6t,3+3{t}^{2})$, $\al''=(-6t,6,6t)$, and $\al'''=(-6,0,6)$. We have $\al'\cdot\al'=18{(1+{t}^{2})}^{2}$, so $\norm{\al'}=3\sqrt{2}(1+{t}^{2})$ and $T=(1-{t}^{2},2t,1+{t}^{2})/(\sqrt{2}(1+{t}^{2}))$. The cross product $\al'\times\al''=(18{t}^{2}-18,-36t,18{t}^{2}+18)=18({t}^{2}-1,-2t,{t}^{2}+1)$ and its norm $\norm{\al'\times\al''}=18\sqrt{2}(1+{t}^{2})$, hence $B=({t}^{2}-1,-2t,{t}^{2}+1)/(\sqrt{2}(1+{t}^{2}))$. Now $N=B\times T=(-2t,1-{t}^{2},0)/(1+{t}^{2})$. By our computation, $\ka=\norm{\al'\times\al''}/{\norm{\al'}}^{3}=1/(3{(1+{t}^{2})}^{2})$ and $\tau=(\al'\times\al'')\cdot\al'''/{\norm{\al'\times\al''}}^{2}=18({t}^{2}-1,-2t,{t}^{2}+1)\cdot 6(-1,0,1)/{(18\sqrt{2}(1+{t}^{2}))}^{2}=216/(648{(1+{t}^{2})}^{2})=1/(3{(1+{t}^{2})}^{2})$.
\end{example}
\begin{definition}
    Let $\al:I\to{\R}^{3}$ be regular. We say $\al$ is a \hdef{cylindrical helix} if there exists $u\in{\R}^{3}$ such that for all $t\in I$, $T(t)\cdot u=\cos\theta$, where $\theta$ is a constant angle.
\end{definition}
\par
If $\al$ has $\norm{\al'}\ne 1$, take the arc-length reparametrization $\wb{\al}$ of $\al$, by our definition, $(T,N,B)$, $\ka$, and $\tau$ are all invariant under this reparametrization, so it suffices to consider a curve with unit speed.
\begin{proposition}
    A regular curve $\al$ on ${\R}^{3}$ with $\ka>0$ is a cylindrical helix if and only if $\tau/\ka$ is constant.
\end{proposition}
\begin{proof}
    Let $\al$ be a curve with $\norm{\al'}=1$. $(\Ra)$ Let $\al$ be a cylindrical helix with unit vector $u$ and constant angle $\theta$, then $T\cdot u=\cos\theta$, so $0=(T\cdot u)'=T'\cdot u=\ka N\cdot u$. Since $\ka>0$, $u\cdot N=0$, then $u=(u\cdot T)T+(u\cdot B)B=\cos\theta T+(u\cdot B)B$. Since $\norm{u}=1$, $u\cdot B=\sin\theta$. Now $0=u'=\cos\theta\ka N-\sin\theta\tau N$, so $\cos\theta\ka=\sin\theta\tau$, which implies $\tau/\ka=\cot\theta$. $(\La)$ Let $\tau/\ka$ be a constant and let $\cot\theta=\tau/\ka$ for some angle $\theta$. Consider $U=\cos\theta T+\sin\theta B$, here $U'=\cos\theta T'+\sin\theta B'=(\cos\theta\ka-\sin\theta\tau)N$, since $\tau/\ka=\cot\theta$, $U'=0$. For all ${s}_{1},{s}_{2}\in I$, $U({s}_{1})=U({s}_{2})$, so pick $U(0)=u$. Now $T\cdot u=T\cdot U=T\cdot(\cos\theta T+\sin\theta B)=\cos\theta$.
\end{proof}
\par
Let $\tau=0$, $\ka>0$, and $\ka'=0$ for some $\al$ with $\norm{\al'}=1$, then $\al$ lies in a circle of radius $1/\ka$ as we proved before. Consider a circle $\al$ in ${\R}^{3}$, define $R:I\to{\R}^{3}$ by $R(s)=\al(s)-c$, where $c$ is the center. We have $(R\cdot R)'=2R'\cdot R=2T\cdot R=0$. Since $T\cdot N=0$, $R=nN$. We have $T=R'=n'N+N'n=n'N+n(-\ka T+\tau B)$, then $(-\ka n-1)T+\tau nB+n'N=0$, so $-\ka n-1=\tau n=n'=0$. Since $n=-1/\ka$, $n\ne 0$, this implies $\tau=0$. Moreover, since $n'=0$, $n$ is a constant, then $\ka$ is a constant.
\begin{definition}
    Let $u\in{\R}^{3}$ be a point with $\norm{u}=1$ and let $V$ be a plane orthogonal to $u$. The \hdef{projection map} of $p\in{\R}^{3}$ onto $V$ is a function $\proj:{\R}^{3}\to V$ defined by $\proj(p)=p-(p\cdot u)u$.
\end{definition}
Let $\al$ be a curve and let $u$ be a unit vector. Let $\be$ be the curve $\be=\proj\comp\al$. Then $\be'=\al'-(\al'\cdot u)u$, $\be''=\al''-(\al''\cdot u)u$, and $\be'''=\al'''-(\al'''\cdot u)u$. We have $\be'\cdot u=\al'\cdot u-(\al'\cdot u)(u\cdot u)=0$, similarly, $\be''\cdot u=0=\be'''\cdot u$, then $(\be'\times\be'')\cdot u=(\be'\cdot u)\be''-(\be''\cdot u)\be'=0$, it suffices to rewrite $\be'\times\be''=nu$ for some $n\in\R$. Now $(\be'\times\be'')\cdot\be'''=nu\cdot\be'''=0$. Hence every curve under a projection map has $\tau=0$.
\begin{definition}
    Let $\al:I\to{\R}^{3}$ be a cylindrical helix with axis direction given by the unit vector $u$. We call $\al$ a \hdef{circular helix} if for every plane orthogonal to $u$, the projection of $\al$ onto that plane is a circle.
\end{definition}
\par
Let $\al$ be a circular helix, then $\tau/\ka=0$ and there exists a unit vector $u$ such that $T\cdot u=\cos\theta$. Consider the projection $\be=\proj\comp\al$ with $\be=\al-(\al\cdot u)\al$.



% 2.5 Covariant Derivatives
\begin{definition}
    Let $W$ be a vector field on ${\R}^{3}$ and let $v\in{T}_{p}({\R}^{3})$ for some $p\in{\R}^{3}$. Then the \hdef{covariant derivative} of $W$ with respect to $v$ is the tangent vector ${\nabla}_{v}W=W(p+tv)'(0)$ at $p$.
\end{definition}
\begin{proposition}
    Let $W=\sum{w}_{i}{U}_{i}$ be a vector field on ${\R}^{3}$ and let $v\in{T}_{p}({\R}^{3})$ for some $p\in{\R}^{3}$. Then ${\nabla}_{v}W=\sum v[{w}_{i}]{U}_{i}(p)$.
\end{proposition}
\begin{proof}
    We have $W(p+tv)=\sum{w}_{i}(p+tv){U}_{i}$. Since $(\d/\d t){w}_{i}(p+tv)$ at $t=0$ is $v[f]$, ${\nabla}_{v}W=\sum v[{w}_{i}]{U}_{i}(p)$.
\end{proof}
\begin{proposition}
    Let $v,w\in{T}_{p}({\R}^{3})$ for some $p\in{\R}^{3}$. Let $Y$ and $Z$ be vector fields on ${\R}^{3}$. For any $a,b\in\R$ and $f:\R\to\R$, the following properties hold.
    \begin{enumerate}
        \item ${\nabla}_{av+bw}Y=a{\nabla}_{v}Y+b{\nabla}_{w}Y$.
        \item ${\nabla}_{v}(aY+bZ)=a{\nabla}_{v}Y+b{\nabla}_{v}Z$.
        \item ${\nabla}_{v}(fY)=v[f]Y(p)+f(p){\nabla}_{v}Y$.
        \item $v[Y\cdot Z]={\nabla}_{v}Y\cdot Z(p)+Y(p)\cdot{\nabla}_{v}Z$.
    \end{enumerate}
\end{proposition}
\begin{proof}
    Rewrite $Y=\sum{y}_{i}{U}_{i}$ and $Z=\sum{z}_{i}{U}_{i}$. (\rom1) We have ${\nabla}_{av+bw}Y=\sum(av+bw)[{y}_{i}]{U}_{i}(p)=a\sum v[{y}_{i}]{U}_{i}+b\sum w[{y}_{i}]{U}_{i}=a{\nabla}_{v}Y+b{\nabla}_{w}Y$. (\rom2) Similarly, ${\nabla}_{v}(aY+bZ)=\sum v[a{y}_{i}+b{z}_{i}]{U}_{i}=a\sum v[{y}_{i}]{U}_{i}+b\sum v[{z}_{i}]{U}_{i}=a{\nabla}_{v}Y+b{\nabla}_{v}Z$. (\rom3) Similarly, ${\nabla}_{v}(fY)=\sum v[f{y}_{i}]{U}_{i}=\sum(v[f]{y}_{i}+fv[{y}_{i}]){U}_{i}=v[f]Y(p)+f(p){\nabla}_{v}Y$. (\rom4) We have $Y\cdot Z=\sum{y}_{i}{z}_{i}$, then $v[Y\cdot Z]=\sum v[{y}_{i}]{z}_{i}{U}_{i}+\sum v[{z}_{i}]{y}_{i}{U}_{i}={\nabla}_{v}Y\cdot Z+Y\cdot{\nabla}_{v}Z$.
\end{proof}
\par
Let $V$ and $W=\sum{w}_{i}$ be vector fields. We define ${\nabla}_{V}W$ at some $p\in{\R}^{3}$ to be ${\nabla}_{V(p)}W$, hence ${\nabla}_{V}W=\sum V[{w}_{i}]{U}_{i}$.
\begin{proposition}
    Let $V$, $W$, $Y$, and $Z$ be vector fields on ${\R}^{3}$. For all functions $f,g:\R\to\R$ and $a,b\in\R$. Then the following properties hold.
    \begin{enumerate}
        \item ${\nabla}_{fV+gW}Y=f{\nabla}_{V}Y+g{\nabla}_{W}Y$.
        \item ${\nabla}_{V}(aY+bZ)=a{\nabla}_{V}Y+b{\nabla}_{V}Z$.
        \item ${\nabla}_{V}(fY)=V[f]Y+f{\nabla}_{V}Y$.
        \item $V[Y\cdot Z]={\nabla}_{V}Y\cdot Z+Y\cdot{\nabla}_{V}Z$.
    \end{enumerate}
\end{proposition}
\begin{proof}
    Since ${\nabla}_{V}W(p)={\nabla}_{V(p)}W$, those properties are direct consequences of the previous proposition.
\end{proof}
% 2.6 Frame Fields
\begin{definition}
    Let ${E}_{1}$, ${E}_{2}$, and ${E}_{3}$ be vector fields on ${\R}^{3}$. We say $\{{E}_{1},{E}_{2},{E}_{3}\}$ is a \hdef{frame field} on ${\R}^{3}$ if ${E}_{i}\cdot{E}_{j}={\de}_{i,j}$.
\end{definition}
\par
If $\{{E}_{1},{E}_{2},{E}_{3}\}$ is a frame field on ${\R}^{3}$, then for all $p\in{\R}^{3}$, $\{{E}_{1}(p),{E}_{2}(p),{E}_{3}(p)\}$ is trivially a frame at $p$.
\begin{example}
    Consider a cylindrical coordinate system with coordinates $(r,\theta,z)$. Define ${E}_{1}=\cos\theta{U}_{1}+\sin\theta{U}_{2}$, ${E}_{2}=-\sin\theta{U}_{1}+\cos\theta{U}_{2}$, and ${E}_{3}={U}_{3}$. For any $p$, ${E}_{1}(p)\cdot{E}_{2}(p)=-\cos\theta\sin\theta+\sin\theta\cos\theta=0$, ${E}_{1}(p)\cdot{E}_{3}(p)=0={E}_{2}(p)\cdot{E}_{3}(p)$, ${E}_{1}(p)\cdot{E}_{1}(p)={\cos}^{2}\theta+{\sin}^{2}\theta={E}_{2}(p)\cdot{E}_{2}(p)=1={E}_{3}(p)\cdot{E}_{3}(p)=1$. Hence $\{{E}_{1},{E}_{2},{E}_{3}\}$ is a frame field on ${\R}^{3}$, known as the \hdef{cylindrical frame field}.
\end{example}
\begin{example}
    Consider a spherical coordinate system with coordinates $(\rho,\theta,\vp)$. Define ${F}_{1}=\cos\vp{E}_{1}+\sin\vp{E}_{3}$, ${F}_{2}={E}_{2}$, and ${F}_{3}=-\sin\vp{E}_{1}+\cos\vp{E}_{3}$, where $\{{E}_{1},{E}_{2},{E}_{3}\}$ is the cylindrical frame field. It is trivial that ${F}_{1}\cdot{F}_{1}={F}_{2}\cdot{F}_{2}={F}_{3}\cdot{F}_{3}=1$. We have 

    Hence $\{{F}_{1},{F}_{2},{F}_{3}\}$ is a frame field on ${\R}^{3}$, called the \hdef{spherical frame field}.
\end{example}
\begin{proposition}
    Let $\{{E}_{1},{E}_{2},{E}_{3}\}$ be a frame field on ${\R}^{3}$. If $V$ is a vector field on ${\R}^{3}$, then $V=\sum{f}_{i}{E}_{i}$, where ${f}_{i}=V\cdot{E}_{i}$. If $V=\sum{f}_{i}{E}_{i}$ and $W={g}_{i}{E}_{i}$, then $V\cdot W=\sum{f}_{i}{g}_{i}$.
\end{proposition}
\begin{proof}
    
\end{proof}
% 2.7 Connection Forms
\newpage
\section{Euclidean Geometry}
\section{Calculus on a Surface}
\section{Shape Operators}
\section{Geometry of Surfaces in ${\R}^{3}$}
\section{Riemannian Geometry}
\section{Global Structure of Surfaces}
\hindex
\end{document}