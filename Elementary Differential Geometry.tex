% chktex-file 1 chktex-file 8 chktex-file 9 chktex-file 12 chktex-file 13 chktex-file 15 chktex-file 17 chktex-file 18 chktex-file 26 chktex-file 31 chktex-file 36 chktex-file 44
\documentclass[10pt]{article}
\input{/Users/mr.hassium/Desktop/Github/Hello-World/hassium.tex}
\def\htitle{Elementary Differential Geometry}
\def\hauthor{Hassium}\let\hfauthor\hauthor
\begin{document}
\hsetup
\htoc
\hmain
\section{Calculus on Euclidean Space}
% 1.1
\begin{definition}
    The \hdef{Euclidean 3-space}, denoted ${\R}^{3}$, is the set of ordered triples of the form $p=({p}_{1},{p}_{2},{p}_{3})$, where ${p}_{i}\in\R$. An element of ${\R}^{3}$ is called a \hdef{point}.
\end{definition}
\par
Let $p=({p}_{1},{p}_{2},{p}_{3}),q=({q}_{1},{q}_{2},{q}_{3})\in{\R}^{3}$ and let $a\in\R$. Define the addition to be $p+q=({p}_{i}+{q}_{i})$ and define the scalar multiplication to be $ap=(a{p}_{i})$. The additive identity $0=(0,0,0)$ is called the \hdef{origin} of ${\R}^{3}$. It is trivial that ${\R}^{3}$ is a vector space over $\R$.
\begin{definition}
    Let $x$, $y$, and $z$ be real-valued functions on ${\R}^{3}$ such that for all $p=({p}_{1},{p}_{2},{p}_{3})\in{\R}^{3}$, $x(p)={p}_{1}$. $y(p)={p}_{2}$, and $z(p)={p}_{3}$. We call $x$, $y$, and $z$ the \hdef{natural coordinate functions} of ${\R}^{3}$.
\end{definition}
\par
Let $x$, $y$, and $z$ be the natural coordinate functions, rewrite $x={x}_{1}$, $y={x}_{2}$, and $z={x}_{3}$. Then we have $p=({p}_{i})=({x}_{i}(p))$.
\begin{definition}
    A real-valued function $f$ on ${\R}^{3}$ is \hdef{differetiable} if all partial derivatives exist and continuous.
\end{definition}
\par
Let $({x}_{1},{x}_{2},{x}_{3}),({y}_{1},{y}_{2},{y}_{3})\in{\R}^{3}$, we define the norm to be $\norm{x-y}=\sqrt{\sum{({x}_{i}-{y}_{i})}^{2}}$.
\begin{definition}
    A subset $O\sub{\R}^{3}$ is \hdef{open} if for all $p\in O$, there exists $\ve>0$ such that $\{x\in{\R}^{3}\mid\norm{x-p}<\ve\}\sub O$. 
\end{definition}
\par
Let $f:O\to\R$ be a function defined on an open set. The differetiability of $f$ at $p$ can be determined entirely from values of $f$ on $O$. This means that differetiation is a local operation. We will discuss this later.
% 1.2
\begin{definition}
    A \hdef{tangent vector} ${v}_{p}$ is an ordered pair ${v}_{p}=(v,p)$, where $v,p\in{\R}^{3}$. Here $v$ is called the \hdef{vector part} and $p$ is called its \hdef{point of application}. Two tangent vectors are said to be \hdef{parallel} if they have the same vector part and different points of application.
\end{definition}
\begin{definition}
    Let $p\in{\R}^{3}$. The \hdef{tangent space} at $p$, denoted ${T}_{p}({\R}^{3})$, is the set of all tangent vectors that have $p$ as point of application.
\end{definition}
\par
Fix a tangent space ${T}_{p}({\R}^{3})$ and let ${T}_{p}({\R}^{3})$ adapt the operations from ${\R}^{3}\times{\R}^{3}$. We have a natural linear map $f:{T}_{p}({\R}^{3})\to{\R}^{3}$ defined by ${v}_{p}\to v$ and it is trivially an isomorphism.
\begin{definition}
    A \hdef{vector field} $V$ on ${\R}^{3}$ is a function $V:{\R}^{3}\to{\R}^{3}$ such that for all $p\in{\R}^{3}$, $V(p)\sub{T}_{p}({\R}^{3})$.
\end{definition}
\par
Let $V$ and $W$ be vector field. Let $f$ be a real-valued function. For all $p\in{\R}^{3}$, define $V+W$ by $(V+W)(p)=V(p)+W(p)$ and $(fV)(p)=f(p)V(p)$.
\begin{definition}
    Let ${U}_{1}$, ${U}_{2}$, and ${U}_{3}$ be vector fields on ${\R}^{3}$ such that ${U}_{1}(p)={(1,0,0)}_{p}$, ${U}_{2}(p)={(0,1,0)}_{p}$, and ${U}_{3}(p)={(0,0,1)}_{p}$ for all $p\in{\R}^{3}$. We call $({U}_{1},{U}_{2},{U}_{3})$ the \hdef{natural frame field} on ${\R}^{3}$.
\end{definition}
\begin{proposition}
    Let $V$ be a vector field on ${\R}^{3}$. There are three uniquely determined real-valued functions ${v}_{1}$, ${v}_{2}$, and ${v}_{3}$ on ${\R}^{3}$ such that $V={v}_{1}{U}_{1}+{v}_{2}{U}_{2}+{v}_{3}{U}_{3}$.
\end{proposition}
\begin{proof}
    For all $p\in{\R}^{3}$, $V(p)={({v}_{1}(p),{v}_{2}(p),{v}_{3}(p))}_{p}={v}_{1}(p){(1,0,0)}_{p}+{v}_{2}(p){(0,1,0)}_{p}+{v}_{3}(p){(0,0,1)}_{p}={v}_{1}(p){U}_{1}(p)+{v}_{2}(p){U}_{2}(p)+{v}_{3}{U}_{3}(p)$, hence $V=\sum{v}_{i}{U}_{i}$.
\end{proof}
\par
The functions ${v}_{1}$, ${v}_{2}$, and ${v}_{3}$ are called the \hdef{Euclidean coordinate functions} on $V$.
\begin{definition}
    A vector field $V$ is \hdef{differetiable} if its Euclidean coordinate functions are differetiable.
\end{definition}
% 1.3
\begin{definition}
    Let $f$ be a differetiable real-valued function on ${\R}^{3}$ and let ${v}_{p}$ be a tangent vector on ${\R}^{3}$. The \hdef{directional derivative} of $f$ with respect to ${v}_{p}$, denoted ${v}_{p}[f]$, is defined to be $(\d/\d t)f(p+tv)$ at $t=0$.
\end{definition}
\begin{remark}
    We will not write the restriction every time for convenience.
\end{remark}
\begin{proposition}
    Let ${v}_{p}={({v}_{1},{v}_{2},{v}_{3})}_{p}$ be a tangent vector, then ${v}_{p}[f]=\sum{v}_{i}(\pa f/\pa{x}_{i})(p)$.
\end{proposition}
\begin{proof}
    Let $p=({p}_{1},{p}_{2},{p}_{3})$. Then ${v}_{p}[f]=(\d/\d t)f(p+tv){\vert}_{t=0}=\sum(\pa f/\pa z)(p)\cdot(\d/\d t)({p}_{i}+t{v}_{i})=\sum(\pa f/\pa{x}_{i})(p){v}_{i}$.
\end{proof}
\begin{example}
    Consider $f={x}^{2}yz$ with $p=(1,1,0)$ and $v=(1,0,-3)$. By the definition, $p+tv=(1+t,1,-3t)$, so ${v}_{p}[f]=(\d/\d t)(-3{t}^{3}-6{t}^{2}-3t)=-3$. Since $(\pa f/\pa x)=2xyz$, $(\pa f/\pa y)={x}^{2}z$, and $(\pa f/\pa z)={x}^{2}y$, we have $(\pa f/\pa x)(p)=(\pa f/\pa y)(p)=0$ and $(\pa f/\pa z)(p)=1$, so ${v}_{p}[f]=-3$.
\end{example}
\begin{proposition}
    Let $f$ and $g$ be functions on ${\R}^{3}$. Let ${v}_{p}$ and ${w}_{p}$ be tangent vectors. For all $a,b\in\R$, the following properties hold.
    \begin{enumerate}
        \item $(a{v}_{p}+b{w}_{p})[f]=a{v}_{p}[f]+b{w}_{p}[f]$.
        \item ${v}_{p}[af+bg]=a{v}_{p}[f]+b{v}_{p}[g]$.
        \item ${v}_{p}[fg]={v}_{p}[f]g(p)+f(p){v}_{p}[g]$.
    \end{enumerate}
\end{proposition}
\begin{proof}
    (\rom1) We have $(a{v}_{p}+b{w}_{p})[f]=\sum(a{v}_{i}+b{w}_{i})(\pa f/\pa{x}_{i})(p)=\sum a{v}_{i}(\pa f/\pa{x}_{i})+\sum b{w}_{i}(\pa f/\pa{x}_{i})(p)=a{v}_{p}[f]+b{w}_{p}[f]$. (\rom2) We have ${v}_{p}[af+bg]=\sum{v}_{i}(\pa(af+bg)/\pa{x}_{i})(p)=\sum{v}_{i}(\pa(af)/\pa{x}_{i})(p)+\sum{v}_{i}(\pa(bg)/\pa{x}_{i})(p)=a{v}_{p}[f]+b{v}_{p}[g]$. (\rom3) We have ${v}_{p}[fg]=\sum{v}_{i}(\pa(fg)/\pa{x}_{i})(p)=\sum{v}_{i}(\pa f/\pa{x}_{i})(p)g(p)+f(p)\sum{v}_{i}(\pa g/\pa{x}_{i})(p)={v}_{p}[f]g(p)+f(p){v}_{p}[g]$.
\end{proof}
\par
Let $V$ be a vecotr field, we define $V[f]$ at $p\in{\R}^{3}$ to be $V(p)[f]$. By the convention, ${U}_{i}(p)[f]=(\pa f/\pa{x}_{i})(p)$.
\begin{proposition}
    Let $V$ and $W$ be vector fields. Let $f$, $g$, and $h$ be real-valued functions. For all $a,b\in\R$, the following properties hold.
    \begin{enumerate}
        \item $(fV+gW)[h]=fV[h]+gW[h]$.
        \item $V[af+bg]=aV[f]+bV[g]$.
        \item $V[fg]=V[f]g+fV[g]$.
    \end{enumerate}
\end{proposition}
\begin{proof}
    (\rom1) For all $p\in{\R}^{3}$, $(fV+gW)(p)[h]=(f(p)V(p)+g(p)W(p))[h]=fV[h]+gW[h]$. (\rom2) For all $p\in{\R}^{3}$, $V(p)[af+bg]=aV(p)[f]+bV(p)[g]$. (\rom3) For all $p\in{\R}^{3}$, $V(p)[f]g(p)+f(p)V(p)[g]=V[f](p)g(p)+f(p)V[g](p)=(V[f]g+fV[g])(p)$.
\end{proof}
\begin{example}
    Let $V=x{U}_{1}-{y}^{2}{U}_{3}$ and let $f={x}^{2}y+{z}^{3}$. Then $V[f]=x{U}_{1}[{x}^{2}y]+x{U}_{1}[{z}^{3}]-{y}^{2}{U}_{3}[{x}^{2}y]-{y}^{2}{U}_{3}[{z}^{3}]=2{x}^{2}y-3{y}^{2}{z}^{2}$.
\end{example}
% 1.4
\par
Let $I\sub\R$ be an open interval. Let $\al:I\to{\R}^{3}$ be a function. We can rewrite $\al(t)$ as $({\al}_{1}(t),{\al}_{2}(t),{\al}_{3}(t))$, where ${\al}_{i}:I\to\R$. We say $\al$ is \hdef{differentiable} if ${\al}_{i}$ are differetiable.
\begin{definition}
    A \hdef{curve} in ${\R}^{3}$ is a differetiable function $\al:I\to{\R}^{3}$, where $I\sub\R$ is an open interval.
\end{definition}
\begin{example}
    Here are some examples of curves.
    \begin{enumerate}
        \item A curve $\al:\R\to{\R}^{3}$ defined by $\al(t)=p+tq$, where $\al(0)=p$ and $q\ne 0$, is called a \hdef{straight line}.
        \item The cruve $\al:\R\to{\R}^{3}$ defined by $\al(t)=(a\cos t,a\sin t,bt)$.
        \item The cruve $\al:\R\to{\R}^{3}$ defined by $\al(t)=(1+\cos t,\sin t,2\sin(t/2))$.
        \item The cruve $\al:\R\to{\R}^{3}$ defined by $\al(t)=({e}^{t},{e}^{-t},\sqrt{2}t)$.
        \item The cruve $\al:\R\to{\R}^{3}$ defined by $\al(t)=(3t-{t}^{3},3{t}^{2},3t+{t}^{3})$.
    \end{enumerate}
\end{example}
\begin{definition}
    Let $\al:I\to{\R}^{3}$ be a curve with $\al=({\al}_{1},{\al}_{2},{\al}_{3})$. For all $t\in I$, the \hdef{velocity vector} of $\al$ at $t$ is the tangent vector $\al'(t)={((\d{\al}_{1}/\d t)(t),(\d{\al}_{2}/\d t)(t),(\d{\al}_{3}/\d t)(t))}_{\al(t)}$ at the point $\al(t)\in{\R}^{3}$. The curve $\al$ is said to be \hdef{regular} if ${\al}_{i}\ne 0$ for all $i$.
\end{definition}
\par
Consider the velocity vector $\al'(t)$, we can rewrite it by the natural frame fields, so $\al'(t)=\sum(\d{\al}_{i}/\d t)(t){U}_{i}(\al(t))$.
\begin{definition}
    Let $\al:I\to{\R}^{3}$ be a curve and let $h:J\to I$ be differetiable, where $J$ is an open interval of $\R$. The \hdef{reparametrization} of $\al$ by $h$ is the composition $\al\comp h:J\to{\R}^{3}$.
\end{definition}
\par
The composition of differentiable functions is differetiable, so any reparametrization is differetiable, which means it is a curve.
\begin{proposition}
    Let $\be$ be the reparametrization of $\al$ by $h$, then $\be'(s)=(\d h/\d s)(s)\al'(h(s))$.
\end{proposition}
\begin{proof}
    Rewrite $\be(s)=\al(h(s))$, then we have $\be'(s)={(\d({\al}_{i}{h}_{i})/\d s)(s)}_{\al(h(s))}={(\d{\al}_{i}/\d s)(h(s))\cdot(\d h/\d s)(s)}_{\al(h(s))}=(\d h/\d s)(s)\al'(h(s))$.
\end{proof}
\begin{proposition}
    Let $\al$ be a curve and let $f$ be a differentiable function on ${\R}^{3}$, then $\al'(t)[f]=(\d(f\al)/\d t)(t)$.
\end{proposition}
\begin{proof}
    We have $\al'(t)[f]=\sum(\d{\al}_{i}/\d t)(t)\cdot(\pa f/\pa{x}_{i})(\al(t))=(\d(f\al)/\d t)(t)$ by the chain rule.
\end{proof}
\par
Now we show a general idea of parametrizations. The proofs will be included in other sections when we have enough tools. Assume every result is correct for now.
\begin{definition}
    Let $f:{\R}^{2}\to\R$ be a differetiable function. The \hdef{level set} ${C}_{f}(a)$ of height $a$ is defined to be the set $\{p\in{\R}^{2}\mid f(p)=a\}$.
\end{definition}
\begin{theorem}[implicit function theorem]
    If $f:{\R}^{2}\to\R$ is continuously differetiable in a neighborhood of some point $({x}_{0},{y}_{0})$ and $\nabla f({x}_{0},{y}_{0})\ne 0$, then there exists a unique differentiable function $\vp$ such that $\vp({x}_{0})={y}_{0}$ and $f(x,\vp(x))=0$ in a neighborhood of ${x}_{0}$.
\end{theorem}

% 1.5
\newpage
\section{Frame Fields}
\section{Euclidean Geometry}
\section{Calculus on a Surface}
\section{Shape Operators}
\section{Geometry of Surfaces in ${\R}^{3}$}
\section{Riemannian Geometry}
\section{Global Structure of Surfaces}
\newpage
\newsection{Exercises and Proofs}
\begin{exercise}[1.1.1]
    Let $f={x}^{2}y$ and $g=y\sin z$ be functions on ${\R}^{3}$. Express the following functions in terms of $x$, $y$, and $z$.
    \begin{enumerate}
        \item $f{g}^{2}$;
        \item $\frac{\pa f}{\pa x}g+\frac{\pa g}{\pa y}f$;
        \item $\frac{{\pa}^{2}(fg)}{\pa y\pa z}$;
        \item $\frac{\pa}{\pa y}(\sin f)$.
    \end{enumerate}
\end{exercise}
\begin{proof}
    (\rom1) We have $f{g}^{2}={x}^{2}y{y}^{2}{\sin}^{2}z={x}^{2}{y}^{3}{\sin}^{2}z$. (\rom2) We have $\frac{\pa f}{\pa x}=2xy$ and $\frac{\pa g}{\pa y}=\sin z$, then $\frac{\pa f}{\pa x}g+\frac{\pa g}{\pa y}f=2x{y}^{2}\sin z+{x}^{2}y\sin z$. (\rom3) We have $fg={x}^{2}{y}^{2}\sin z$, then $\frac{{\pa}^{2}(fg)}{\pa y\pa z}=2{x}^{2}y\cos z$. (\rom4) We have $\sin f=\sin({x}^{2}y)$, then $\frac{\pa}{\pa y}(\sin f)={x}^{2}\cos({x}^{2}y)$.
\end{proof}
\begin{exercise}[1.1.3]
    Express $\frac{\pa f}{\pa x}$ in terms of $x$, $y$, and $z$ for the following functions.
    \begin{enumerate}
        \item $f=x\sin(xy)+y\cos(xz)$;
        \item $f=\sin g$, $g={e}^{h}$, and $h={x}^{2}+{y}^{2}+{z}^{2}$.
    \end{enumerate}
\end{exercise}
\begin{proof}
    (\rom1) We have $\frac{\pa f}{\pa x}=\frac{x\sin(xy)}{\pa x}+\frac{\pa y\cos(xz)}{\pa x}=\sin(xy)+xy\cos(xy)-yz\sin(xz)$. (\rom2) We have $f=\sin({e}^{{x}^{2}+{y}^{2}+{z}^{2}})$, then $\frac{\pa f}{\pa x}=2x\cos({e}^{{x}^{2}+{y}^{2}+{z}^{2}}){e}^{{x}^{2}+{y}^{2}+{z}^{2}}$.
\end{proof}
\begin{exercise}[1.2.1]
    Let $v=(-2,1,-1)$ and $w=(0,1,3)$. At an arbitrary point $p$, express the tangent vector $3{v}_{p}-2{w}_{p}$ as a linear combination of ${U}_{1}(p)$, ${U}_{2}(p)$, and ${U}_{3}(p)$.
\end{exercise}
\hindex
\end{document}