% chktex-file 1 chktex-file 8 chktex-file 9 chktex-file 12 chktex-file 13 chktex-file 15 chktex-file 17 chktex-file 18 chktex-file 26 chktex-file 31 chktex-file 36 chktex-file 44
\documentclass[10pt]{article}
\input{/Users/mr.hassium/Desktop/Github/Hello-World/hassium.tex}
\def\htitle{Elementary Differential Geometry}
\def\hauthor{Hassium}\let\hfauthor\hauthor
\begin{document}
\hsetup
\htoc
\hmain
\section{Calculus on Euclidean Space}
\begin{definition}
    The \hdef{Euclidean 3-space}, denoted ${\R}^{3}$, is the set of ordered triples of the form $p=({p}_{1},{p}_{2},{p}_{3})$, where ${p}_{i}\in\R$. An element of ${\R}^{3}$ is called a \hdef{point}.
\end{definition}
\par
Let $p=({p}_{1},{p}_{2},{p}_{3}),q=({q}_{1},{q}_{2},{q}_{3})\in{\R}^{3}$ and let $a\in\R$. Define the addition to be $p+q=({p}_{i}+{q}_{i})$ and define the scalar multiplication to be $ap=(a{p}_{i})$. The additive identity $0=(0,0,0)$ is called the \hdef{origin} of ${\R}^{3}$. It is trivial that ${\R}^{3}$ is a vector space over $\R$.
\begin{definition}
    Let $x$, $y$, and $z$ be real-valued functions on ${\R}^{3}$ such that for all $p=({p}_{1},{p}_{2},{p}_{3})\in{\R}^{3}$, $x(p)={p}_{1}$. $y(p)={p}_{2}$, and $z(p)={p}_{3}$. We call $x$, $y$, and $z$ the \hdef{natural coordinate functions} of ${\R}^{3}$.
\end{definition}
\par
Let $x$, $y$, and $z$ be the natural coordinate functions, rewrite $x={x}_{1}$, $y={x}_{2}$, and $z={x}_{3}$. Then we have $p=({p}_{i})=({x}_{i}(p))$.
\begin{definition}
    A real-valued function $f$ on ${\R}^{3}$ is \hdef{differetiable} if all partial derivatives exist and continuous.
\end{definition}
\begin{definition}
    A subset $O\sub{\R}^{3}$ is \hdef{open} if for all $p\in O$, there exists $\ve>0$ such that $\{x\in{\R}^{3}\mid\norm{x-p}<\ve\}\sub O$. 
\end{definition}
\par
Let $f:O\to\R$ be a function defined on an open set. The differetiability of $f$ at $p$ can be determined entirely from values of $f$ on $O$. This means that differetiation is a local operation. We will give a proof of this later.
\begin{definition}
    A \hdef{tangent vector} ${v}_{p}$ is an ordered pair ${v}_{p}=(v,p)$, where $v,p\in{\R}^{3}$. Here $v$ is called the \hdef{vector part} and $p$ is called its \hdef{point of application}. Two tangent vectors are said to be \hdef{parallel} if they have the same vector part and different points of application.
\end{definition}
\begin{definition}
    Let $p\in{\R}^{3}$. The \hdef{tangent space} at $p$, denoted ${T}_{p}({\R}^{3})$, is the set of all tangent vectors that have $p$ as point of application.
\end{definition}




\newpage
\section{Frame Fields}
\section{Euclidean Geometry}
\section{Calculus on a Surface}
\section{Shape Operators}
\section{Geometry of Surfaces in ${\R}^{3}$}
\section{Riemannian Geometry}
\section{Global Structure of Surfaces}
\newpage
\newsection{Exercises and Proofs}
\begin{exercise}[1.1.1]
    Let $f={x}^{2}y$ and $g=y\sin z$ be functions on ${\R}^{3}$. Express the following functions in terms of $x$, $y$, and $z$.
    \begin{enumerate}
        \item $f{g}^{2}$;
        \item $\frac{\pa f}{\pa x}g+\frac{\pa g}{\pa y}f$;
        \item $\frac{{\pa}^{2}(fg)}{\pa y\pa z}$;
        \item $\frac{\pa}{\pa y}(\sin f)$.
    \end{enumerate}
\end{exercise}
\begin{proof}
    (\rom1) We have $f{g}^{2}={x}^{2}y{y}^{2}{\sin}^{2}z={x}^{2}{y}^{3}{\sin}^{2}z$. (\rom2) We have $\frac{\pa f}{\pa x}=2xy$ and $\frac{\pa g}{\pa y}=\sin z$, then $\frac{\pa f}{\pa x}g+\frac{\pa g}{\pa y}f=2x{y}^{2}\sin z+{x}^{2}y\sin z$. (\rom3) We have $fg={x}^{2}{y}^{2}\sin z$, then $\frac{{\pa}^{2}(fg)}{\pa y\pa z}=2{x}^{2}y\cos z$. (\rom4) We have $\sin f=\sin({x}^{2}y)$, then $\frac{\pa}{\pa y}(\sin f)={x}^{2}\cos({x}^{2}y)$.
\end{proof}
\begin{exercise}[1.1.3]
    Express $\frac{\pa f}{\pa x}$ in terms of $x$, $y$, and $z$ for the following functions.
    \begin{enumerate}
        \item $f=x\sin(xy)+y\cos(xz)$;
        \item $f=\sin g$, $g={e}^{h}$, and $h={x}^{2}+{y}^{2}+{z}^{2}$.
    \end{enumerate}
\end{exercise}
\begin{proof}
    (\rom1) We have $\frac{\pa f}{\pa x}=\frac{x\sin(xy)}{\pa x}+\frac{\pa y\cos(xz)}{\pa x}=\sin(xy)+xy\cos(xy)-yz\sin(xz)$. (\rom2) We have $f=\sin({e}^{{x}^{2}+{y}^{2}+{z}^{2}})$, then $\frac{\pa f}{\pa x}=2x\cos({e}^{{x}^{2}+{y}^{2}+{z}^{2}}){e}^{{x}^{2}+{y}^{2}+{z}^{2}}$.
\end{proof}
\begin{exercise}[1.2.1]
    Let $v=(-2,1,-1)$ and $w=(0,1,3)$. At an arbitrary point $p$, express the tangent vector $3{v}_{p}-2{w}_{p}$ as a linear combination of ${U}_{1}(p)$, ${U}_{2}(p)$, and ${U}_{3}(p)$.
\end{exercise}
\hindex
\end{document}