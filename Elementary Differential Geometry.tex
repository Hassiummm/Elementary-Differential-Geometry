% chktex-file 1 chktex-file 8 chktex-file 9 chktex-file 12 chktex-file 13 chktex-file 15 chktex-file 17 chktex-file 18 chktex-file 26 chktex-file 31 chktex-file 36 chktex-file 44
\documentclass[10pt]{article}
\input{/Users/mr.hassium/Desktop/Github/Hello-World/hassium.tex}
\def\htitle{Elementary Differential Geometry}
\def\hauthor{Hassium}\let\hfauthor\hauthor
\begin{document}
\hsetup
\htoc
\hmain
\section{Calculus on Euclidean Space}
\begin{definition}
    The \hdef{Euclidean 3-space}, denoted ${\R}^{3}$, is the set of ordered triples of the form $p=({p}_{1},{p}_{2},{p}_{3})$, where ${p}_{i}\in\R$. An element of ${\R}^{3}$ is called a \hdef{point}.
\end{definition}
\par
Let $p=({p}_{1},{p}_{2},{p}_{3}),q=({q}_{1},{q}_{2},{q}_{3})\in{\R}^{3}$ and let $a\in\R$. Define the addition to be $p+q=({p}_{i}+{q}_{i})$ and define the scalar multiplication to be $ap=(a{p}_{i})$. The additive identity $0=(0,0,0)$ is called the \hdef{origin} of ${\R}^{3}$. It is trivial that ${\R}^{3}$ is a vector space over $\R$.
\begin{definition}
    Let $x$, $y$, and $z$ be real-valued functions on ${\R}^{3}$ such that for all $p=({p}_{1},{p}_{2},{p}_{3})\in{\R}^{3}$, $x(p)={p}_{1}$. $y(p)={p}_{2}$, and $z(p)={p}_{3}$. We call $x$, $y$, and $z$ the \hdef{natural coordinate functions} of ${\R}^{3}$.
\end{definition}
\par
Let $x$, $y$, and $z$ be the natural coordinate functions, rewrite $x={x}_{1}$, $y={x}_{2}$, and $z={x}_{3}$. Then we have $p=({p}_{i})=({x}_{i}(p))$.
\begin{definition}
    A real-valued function $f$ on ${\R}^{3}$ is \hdef{differetiable} if all partial derivatives exist and continuous.
\end{definition}
\begin{definition}
    A subset $O\sub{\R}^{3}$ is \hdef{open} if for all $p\in O$, there exists $\ve>0$ such that $\{x\in{\R}^{3}\mid\norm{x-p}<\ve\}\sub O$. 
\end{definition}
\par
Let $f:O\to\R$ be a function defined on an open set. The differetiability of $f$ at $p$ can be determined entirely from values of $f$ on $O$. This means that differetiation is a local operation. We will give a proof of this later.
\begin{definition}
    A \hdef{tangent vector} ${v}_{p}$ is an ordered pair ${v}_{p}=(v,p)$, where $v,p\in{\R}^{3}$. Here $v$ is called the \hdef{vector part} and $p$ is called its \hdef{point of application}. Two tangent vectors are said to be \hdef{parallel} if they have the same vector part and different points of application.
\end{definition}
\begin{definition}
    Let $p\in{\R}^{3}$. The \hdef{tangent space} at $p$, denoted ${T}_{p}({\R}^{3})$, is the set of all tangent vectors that have $p$ as point of application.
\end{definition}




\newpage
\section{Frame Fields}
\section{Euclidean Geometry}
\section{Calculus on a Surface}
\section{Shape Operators}
\section{Geometry of Surfaces in ${\R}^{3}$}
\section{Riemannian Geometry}
\section{Global Structure of Surfaces}
\newpage
\newsection{Exercises and Proofs}
\begin{exercise}[1.1.1]
    Let $f={x}^{2}y$ and $g=y\sin z$ be functions on ${\R}^{3}$. Express the following functions in terms of $x$, $y$, and $z$: $f{g}^{2}$; $(\pa f/\pa x)g+(\pa g/\pa y)f$; ${\pa}^{2}(fg)/(\pa y\pa z)$; $(\pa/\pa y)\sin f$.
\end{exercise}
\begin{proof}
    (\rom1) We have $f{g}^{2}={x}^{2}y{y}^{2}{\sin}^{2}z={x}^{2}{y}^{3}{\sin}^{2}z$. (\rom2) We have $\pa f/\pa x=2xy$ and $\pa g/\pa y=\sin z$, then $(\pa f/\pa x)g+(\pa g/\pa y)f=2x{y}^{2}\sin z+{x}^{2}y\sin z$. (\rom3) We have $fg={x}^{2}{y}^{2}\sin z$, then ${\pa}^{2}(fg)/(\pa y\pa z)=2{x}^{2}y\cos z$. (\rom4) We have $\sin f=\sin({x}^{2}y)$, then $(\pa/\pa y)(\sin f)={x}^{2}\cos({x}^{2}y)$.
\end{proof}
\begin{exercise}[1.1.3]
    Express $\pa f/\pa x$ in terms of $x$, $y$, and $z$ for the following functions.
    \begin{enumerate}
        \item $f=x\sin(xy)+y\cos(xz)$;
        \item $f=\sin g$, $g={e}^{h}$, and $h={x}^{2}+{y}^{2}+{z}^{2}$.
    \end{enumerate}
\end{exercise}
\begin{proof}
    (\rom1) We have $(\pa f/\pa x)=\pa(x\sin(xy))/\pa x+\pa(y\cos(xz))/\pa x=\sin(xy)+xy\cos(xy)-yz\sin(xz)$. (\rom2) We have $f=\sin({e}^{{x}^{2}+{y}^{2}+{z}^{2}})$, then $(\pa f/\pa x)=2x\cos({e}^{{x}^{2}+{y}^{2}+{z}^{2}}){e}^{{x}^{2}+{y}^{2}+{z}^{2}}$.
\end{proof}



\begin{exercise}[1.2.1]
    Let $v=(-2,1,-1)$ and $w=(0,1,3)$. At an arbitrary point $p$, express the tangent vector $3{v}_{p}-2{w}_{p}$ as a linear combination of ${U}_{1}(p)$, ${U}_{2}(p)$, and ${U}_{3}(p)$.
\end{exercise}
\begin{proof}
    We have $3{v}_{p}-2{w}_{p}={(-6,1,-9)}_{p}=-6{U}_{1}(p)+{U}_{2}(p)-9{U}_{3}(p)$.
\end{proof}


\begin{exercise}[1.2.3]
    Let $p=({p}_{1},{p}_{2},{p}_{3})$. In each case, express the given vector field $V$ in the standard form $\sum{v}_{i}{U}_{i}$. 
    \begin{enumerate}
        \item $2{z}^{2}{U}_{1}=7V+xy{U}_{3}$.
        \item $V(p)={({p}_{1},{p}_{3}-{p}_{1},0)}_{p}$ for all $p$.
        \item $V=2(x{U}_{1}+y{U}_{2})-x({U}_{1}-{y}^{2}{U}_{3})$.
        \item For all $p\in{\R}^{3}$, $V(p)$ is the vector from $({p}_{1},{p}_{2},{p}_{3})$ to $(1+{p}_{1},{p}_{2}{p}_{3},{p}_{2})$.
        \item For all $p\in{\R}^{3}$, $V(p)$ is the vector from $p$ to 0.
    \end{enumerate}
\end{exercise}
\begin{proof}
    (\rom1) We have $V=(2{z}^{2}{U}_{1}-xy{U}_{3})/7$. For all $p\in{\R}^{3}$, $V(p)=((2{z}^{2},0,0)-(0,0,xy))/7=(2{z}^{2}/7,0,-xy/7)$, so $({v}_{i})=(2{z}^{2}/7,0,-xy/7)$. (\rom2) Here $V(p)=x{U}_{1}+(z-x){U}_{2}+0{U}_{3}$.
\end{proof}
\begin{exercise}[1.2.5]
    Let ${V}_{1}={U}_{1}-x{U}_{3}$, ${V}_{2}={U}_{2}$, and ${V}_{3}=x{U}_{1}+{U}_{3}$. Prove that the vectors ${V}_{1}(p)$, ${V}_{2}(p)$, ${V}_{3}(p)$ are linearly indepnedent at each $p\in{\R}^{3}$. Express the vector field $x{U}_{1}+y{U}_{2}+z{U}_{3}$ as a linear combination of ${V}_{i}$.
\end{exercise}
\begin{proof}
    For all $p\in{\R}^{3}$, we have ${V}_{1}(p)={U}_{1}(p)-x{U}_{3}(p)=(1,0,-x)$. Similarly, ${V}_{2}(p)=(0,1,0)$ and ${V}_{3}=(x,0,1)$. Consider $a{V}_{1}(p)+b{V}_{2}(p)+c{V}_{3}(p)=0$, where $a,b,c\in\R$. Solve for $(a,b,c)$, then $c({x}^{2}+1)=0$, so $c=0$. Now $(a,b,c)=(0,0,0)$, hence ${V}_{i}(p)$ are linearly independent. For all $p\in{\R}^{3}$, $a{V}_{1}(p)+b{V}_{2}(p)+c{V}_{3}(p)=(a+cx,b,c-a)$ and $x{U}_{1}(p)+y{U}_{2}(p)+z{U}_{3}(p)=(x,y,z)$. Solve $(a+cx,b,c-a)=(x,y,z)$, then $(a,b,c)=((x-zx)/(1+{x}^{2}),y,({x}^{2}+z)/(1+{x}^{2}))$. 
\end{proof}
\begin{exercise}[1.3.1]
    Let ${v}_{p}$ be the tangent vector with $v=(2,-1,3)$ and $p=(2,0,-1)$. Use the definition to compute the directional derivative for the following functions: $f={y}^{2}z$; $f={x}^{7}$; $f={e}^{x}\cos y$.
\end{exercise}
\begin{proof}
    We have $p+tv=(2+2t,-t,3t-1)$. (\rom1) Now $f(p+tv)=3{t}^{3}-{t}^{2}$, then ${v}_{p}[f]=9{t}^{2}-2t=0$. (\rom2) Now $f(p+tv)={(2+2t)}^{7}$, then ${v}_{p}[f]=7{(2+2t)}^{6}\cdot 2=14{(2+2t)}^{6}=7\cdot{2}^{7}$. (\rom3) Now $f(p+tv)={e}^{2+2t}\cos(-t)$, then ${v}_{p}[f]={e}^{2+2t}\sin(-t)+2{e}^{2+2t}\cos(-t)=2{e}^{2}$.
\end{proof}
\begin{exercise}[1.3.3]
    Let $V={y}^{2}{U}_{1}-x{U}_{3}$. Let $f=xy$ and let $g={z}^{3}$. Compute the following functions: $V[f]$; $V[g]$; $V[fg]$; $fV[g]-gV[f]$; $V[{f}^{2}+{g}^{2}]$; $V[V[f]]$.
\end{exercise}
\begin{proof}
    (\rom1) We have $V[f]={y}^{2}{U}_{1}[xy]-x{U}_{3}[xy]={y}^{3}$. (\rom2) We have $V[g]={y}^{2}{U}_{1}[{z}^{3}]-x{U}_{3}[{z}^{3}]=-3x{z}^{2}$. (\rom3) We have $V[fg]=V[f]g+fV[g]={y}^{3}{z}^{3}-3{x}^{2}y{z}^{2}$. (\rom4) We have $fV[g]-gV[f]=-3{x}^{2}y{z}^{2}-{y}^{3}{z}^{3}$. (\rom5) We have $V[{f}^{2}+{g}^{2}]=V[{f}^{2}]+V[{g}^{2}]=V[f]f+fV[f]+V[g]g+gV[g]=2x{y}^{4}-6x{z}^{5}$. (\rom6) We have $V[V[f]]=V[{y}^{3}]={y}^{2}{U}_{1}[{y}^{3}]-x{U}_{3}[{y}^{3}]=0$.
\end{proof}
\begin{exercise}[1.3.5]
    If $V[f]=W[f]$ for all $f$ on ${\R}^{3}$, prove that $V=W$.
\end{exercise}
\begin{proof}
    Let $V=\sum{a}_{i}{U}_{i}$ and let $W=\sum{b}_{i}{U}_{i}$. Since $V[f]=W[f]$, $(V-W)[f]=\sum({a}_{i}-{b}_{i})(\pa f/\pa{x}_{i})=0$. Pick $f=x$, then ${a}_{1}={b}_{1}$. Similarly, if we pick $f=y$ and $f=z$, we have ${a}_{2}={b}_{2}$ and ${a}_{3}={b}_{3}$. Hence $V=W$.
\end{proof}

\begin{exercise}[1.4.1]
    Compute the velocity vector of the curve $\al(t)=(1+\cos t,\sin t,2\sin(t/2))$ for arbitrary $t$ and for $t=0$, $t=\pi/2$, $t=\pi$. 
\end{exercise}

\begin{exercise}[1.4.3]
    Find the coordinate functions of the curve $\be=\al(h)$, where $\al(t)=(1+\cos t,\sin t,2\sin(t/2))$ and $h(s)=\inv{\cos}(s)$ on $(0,1)$.
\end{exercise}

\begin{exercise}[1.4.5]
    Find the equation of the straight line through the points $(1,-3,-1)$ and $(6,2,1)$. Does this line meet the line through the points $(-1,1,0)$ and $(-5,-1,-1)$?
\end{exercise}

\begin{exercise}[1.4.7]
    Show that the curves with coordinate functions $(t,1+{t}^{2},t)$, $(\sin t,\cos t,t)$, and $(\sinh t,\cosh t,t)$ all have the same initial velocity. Let $f={x}^{2}-{y}^{2}+{z}^{2}$, compute ${v}_{p}[f]$ by calculating $\d(f(a))/\d t{\vert}_{t=0}$ using each of the three curves above.
\end{exercise}

\begin{exercise}[1.4.9]
    For a fixed $t$, the tangent line to a regular curve $\al$ at the point $\al(t)$ is the straight line $u\mt\al(t)+u\al'(t)$. Find the tangent line to the helix $\al(t)=(2\cos t, 2\sin t,t)$ at the points $\al(0)$ and $\al(\pi/4)$.
\end{exercise}




\begin{exercise}[1.7.9]
    Let $F:{\R}^{n}\to{\R}^{m}$ and $G:{\R}^{m}\to{\R}^{p}$ be mappings. Prove $GF$ is a differentiable mapping. Prove $(GF)*=G*F*$. If $F$ is a diffeomorphism, then so is its inverse mapping $\inv{F}$.
\end{exercise}
\begin{proof}
    
\end{proof}
\hindex
\end{document}